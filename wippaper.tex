% \documentclass[12pt,twocolumn]{article}
% Copernicus stuff
\documentclass[gmd,manuscript]{copernicus}
%\documentclass[gmd,manuscript]{../171128_Copernicus_LaTeX_Package/copernicus} %durack

% page/line labeling and referencing
% from http://goo.gl/HvS9BK
\newcommand{\pllabel}[1]{\label{p-#1}\linelabel{l-#1}}
\newcommand{\plref}[1]{see page~\pageref{p-#1}, line~\lineref{l-#1}.}
% answer environment for reviewer responses
\newenvironment{answer}{\color{blue}}{}
\usepackage{enumitem}

\hypersetup{colorlinks=true,urlcolor=blue,citecolor=red}
% \hypersetup{colorlinks=false}
% \newcommand{\degree}{\ensuremath{^\circ}}
% \newcommand{\order}{\ensuremath{\mathcal{O}}}
% \newcommand{\bibref}[1] { \cite{ref:#1}}
% \newcommand{\pipref}[1] {\citep{ref:#1}}
% \newcommand{\ceqref}[1] {\mbox{CodeBlock \ref{code:#1}}}
% \newcommand{\charef}[1] {\mbox{Chapter \ref{cha:#1}}}
% \newcommand{\eqnref}[1] {\mbox{Eq.     \ref{eq:#1}}}
% \newcommand{\figref}[1] {\mbox{Figure   \ref{fig:#1}}}
% \newcommand{\secref}[1] {\mbox{Section  \ref{sec:#1}}}
% \newcommand{\appref}[1] {\mbox{Appendix \ref{sec:#1}}}
% \newcommand{\tabref}[1] {\mbox{Table   \ref{tab:#1}}}
\newcommand{\urlref}[2] {\href{#1}{#2}\footnote{\url{#1}, retrieved \today.}}

\newcommand{\editorial}[1]{\protect{\color{red}#1}}

\runningtitle{WIP Paper Draft \today}
\runningauthor{Balaji et al.}

\begin{document}

\title{Requirements for a global data infrastructure in support of CMIP6}

\Author[1,2]{Venkatramani}{Balaji}
\Author[3]{Karl E.}{Taylor}
\Author[4]{Martin}{Juckes}
\Author[5,4]{Bryan N.}{Lawrence}
\Author[6]{Michael}{Lautenschlager}
\Author[7,2]{Chris}{Blanton}
\Author[8]{Luca}{Cinquini}
\Author[9]{S\'ebastien}{Denvil}
\Author[3]{Paul J.}{Durack}
\Author[10]{Mark}{Elkington}
\Author[9]{Francesca}{Guglielmo}
\Author[9,4]{Eric}{Guilyardi}
\Author[4]{David}{Hassell}
\Author[11]{Slava}{Kharin}
\Author[6]{Stefan}{Kindermann}
\Author[1,2]{Sergey}{Nikonov}
\Author[7,2]{Aparna}{Radhakrishnan}
\Author[6]{Martina}{Stockhause}
\Author[6]{Tobias}{Weigel}
\Author[3]{Dean}{Williams}


\affil[1]{Princeton University, Cooperative Institute of Climate
  Science, Princeton NJ, USA}
\affil[2]{NOAA/Geophysical Fluid Dynamics Laboratory, Princeton NJ,
  USA}
\affil[3]{PCMDI, Lawrence Livermore National Laboratory, Livermore, CA, USA}
\affil[4]{Science and Technology Facilities Council, Abingdon, UK}
\affil[5]{National Center for Atmospheric Science and University of
  Reading, UK}
\affil[6]{Deutsches KlimaRechenZentrum GmbH, Hamburg, Germany}
\affil[7]{Engility Inc., NJ, USA}
\affil[8]{Jet Propulsion Laboratory (JPL), 4800 Oak Grove Drive,
Pasadena, CA 91109, USA}
\affil[9]{Institut Pierre-Simon Laplace, CNRS/UPMC, Paris, France}
\affil[10]{Met Office, FitzRoy Road, Exeter, EX1 3PB, UK}
\affil[11]{Canadian Centre for Climate Modelling and Analysis, Atmospheric Environment Service, University of Victoria, BC, Canada}
% \affil[10]{NCAR}

\correspondence{V. Balaji (\texttt{balaji@princeton.edu})}

\received{}
\pubdiscuss{} %% only important for two-stage journals
\revised{}
\accepted{}
\published{}

%% These dates will be inserted by Copernicus Publications during the typesetting process.


\firstpage{1}

\maketitle

% \pagebreak
\abstract{The World Climate Research Programme (WCRP)'s Working Group
  on Climate Modeling (WGCM) Infrastructure Panel (WIP) was formed in
  2014 in response to the explosive growth in size and complexity of
  Coupled Model Intercomparison Projects (CMIPs) between CMIP3
  (2005-06) and CMIP5 (2011-12). This article presents the WIP
  recommendations for the global data infrastructure needed to support
  CMIP design, future growth and evolution. Developed in close
  coordination with those who build and run the existing
  infrastructure (the Earth System Grid Federation), the
  recommendations are based on several principles beginning with the
  need to separate requirements, implementation, and operations. Other
  important principles include the consideration of
  \pllabel{RC2-2}
  the diversity of community needs around data -- a \emph{data
    ecosystem} -- the importance of provenance, the need for
  automation, and the obligation to measure costs and benefits.
  
  This paper concentrates on requirements, recognising the diversity
  of communities involved (modelers, analysts, software developers, 
  and downstream users). Such requirements include the need for 
  scientific reproducibility and accountability alongside the need 
  to record and track data usage.
\pllabel{RC1-1}
  One key element is to generate a dataset-centric rather than
  system-centric focus, with an aim to making the infrastructure less
  prone to systemic failure.

  With these overarching principles and requirements, the WIP has
  produced a set of position papers, which are summarized here. They
  provide specifications for managing and delivering model output,
  including strategies for replication and versioning, licensing, data
  quality assurance, citation, long-term archival, and dataset
  tracking. They also describe a new and more formal approach for
  specifying what data, and associated metadata, should be saved,
  which enables future data volumes to be estimated.
 
  The paper concludes with a future-facing consideration of the global
  data infrastructure evolution that follows from the blurring of
  boundaries between climate and weather, and the changing nature of
  published scientific results in the digital age. }
% \pagebreak

\introduction
\label{sec:intro}

CMIP6 \citep{ref:eyringetal2016a}, the latest Coupled Model
Intercomparison Project (CMIP), can trace its genealogy back to the
Charney Report \citep{ref:charneyetal1979}. This seminal report on the
links between CO$_2$ and climate was an authoritative summary of the
state of the science at the time, and produced findings that have
stood the test of time \citep{ref:bonyetal2013}. It is often noted
\citep[see, e.g][]{ref:andrewsetal2012}
\pllabel{RC1-2}
that the range and uncertainty bounds on equilibrium climate
sensitivity generated in this report have not fundamentally changed,
despite the enormous increase in resources devoted to analysing the
problem in decades since.

Beyond its
\pllabel{RC2-4}
enduring findings on climate sensitivity, the Charney Report also gave
rise to a methodology for the treatment of uncertainties and gaps in
understanding, which has been equally influential, and is in fact the
basis of CMIP itself. The Report can be seen as one of the first uses
of the \emph{multi-model ensemble}. At the time, there were two models
available \pllabel{RC1-3} representing the equilibrium response of the
climate system to a change in CO$_2$ forcing, one from Syukuro
Manabe's group at NOAA's Geophysical Fluid Dynamics Laboratory, and
the other from James Hansen's group at NASA's Goddard Institute for
Space Studies. Then as now, these groups marshaled vast
state-of-the-art computing and data resources to run very challenging
simulations of the Earth system. The Report's results were based on an
ensemble of
\pllabel{RC2-5}
three runs from the Manabe group, \pllabel{RC1-4} labeled M1-M3, and two
from the Hansen group, \pllabel{RC1-5} labeled H1-H2.

The Atmospheric Model Intercomparison Project
\citep[AMIP:][]{ref:gates1992} was one of the first systematic
cross-model comparisons open to anyone who wished to participate.
\pllabel{RC1-6}
By the time of the Inter-Governmental Panel on Climate Change (IPCC)'s
First Assessment Report (FAR) in 1990 \citep{ref:houghtonetal1992},
\pllabel{RC1-9}
the process had been formalized. At this stage, there were
\pllabel{RC2-6}
five models participating in the exercise, and some of what
\pllabel{RC2-7}
is now called the ``Diagnosis, Evaluation, and Characterization of
Klima'' \citep[DECK, see][]{ref:eyringetal2016a}
experiments\footnote{``Klima'' is German for ``climate''.} had been
standardized (AMIP, a pre-industrial control, 1\% per year CO$_2$
increase to doubling, etc). The ``scenarios'' had emerged as well, for
a total of
\pllabel{RC2-6b}
five different experimental protocols. Fast-forwarding to today, CMIP6
expects more than 75 models from around 35 modeling centers \citep[in
14 countries, a stark contrast to the US monopoly
in][]{ref:charneyetal1979} to participate in the DECK and historical
experiments \citep[Table~2 of][]{ref:eyringetal2016a}, and some subset
of these to participate in one or more the 21 MIPs endorsed by the
CMIP Panel \citep[Table~3 of][, now 23 with two new endorsed MIPs
since]{ref:eyringetal2016a}. \pllabel{RC1-7} The MIPs call for over
200 experiments, a considerable expansion over CMIP5.

Alongside the experiments themselves is the data request which
defines, for each CMIP experiment, what output each model should
provide for analysis. The complexity of this data request has also
grown tremendously over the CMIP era. A typical dataset from the FAR
archive (\urlref{https://goo.gl/M1WSJy}{from the GFDL R15 model}) lists
climatologies and time series of two variables, and the dataset size
is about 200~MB. The CMIP6 Data Request \cite{ref:juckesetal2015}
lists literally thousands of variables from the hundreds of
experiments mentioned above. This growth in complexity is testament to
the modern understanding of many physical, chemical and biological
processes which were simply absent from the Charney Report era models.

The simulation output is now a primary scientific resource for
researchers the world over, rivaling the volume of observed weather
and climate data from the global array of sensors and satellites
\citep{ref:overpecketal2011}. Climate science, and observed and simulated
climate data in particular, have now become primary elements in the
``vast machine'' \citep{ref:edwards2010} serving the global climate and
weather enterprise.
% It could be worthwhile to quantify (in $USD) the impact, as forecasting
% in particular has yielded considerable social and economic gains

Managing and sharing this huge amount of data is an enterprise in its
own right -- and the solution established for CMIP5 was the global
Earth System Grid Federation
\citep[ESGF,][]{ref:williamsetal2011a,ref:williamsetal2015}. ESGF was
identified by the WCRP Joint Scientific Committee in 2013 as the
recommended infrastructure for data archiving and dissemination for
the Programme.
\pllabel{RC2-12}
A map of sites participating in the ESGF are shown in
\pllabel{RC2-8}
Figure~\ref{fig:esgf} drawn from
\urlref{https://portal.enes.org/data/is-enes-data-infrastructure/esgf}{IS-ENES
  Data Portal}. The sites are diverse and responsive to many national
and institutional missions. With multiple agencies and institutions,
and many uncoordinated and possibly conflicting requirements, the ESGF
itself is a complex and delicate
\pllabel{RC2-10}
artifact to manage.

\begin{figure*}
  \begin{center}
    \includegraphics[width=175mm]{images/esgf-map-2017.png}
  \end{center}
  \caption{Sites participating in the Earth System Grid Federation in
    May 2017. Figure courtesy IS-ENES Data Portal. }
  \label{fig:esgf}
\end{figure*}

The sheer size and complexity of this infrastructure emerged as a
matter of great concern at the end of CMIP5, when the growth in data
volume relative to CMIP3 (from 40~TB to 2~PB, a 50-fold increase in 6
years) suggested the community was on an unsustainable path. These
concerns led to the 2014 recommendation of the WGCM to form an
\emph{infrastructure panel} (based upon
\pllabel{RC2-11}
\urlref{https://goo.gl/FHqbNN}{a proposal} at the 2013 annual
meeting). The WGCM Infrastructure Panel (WIP) was tasked with
examining the global computational and data infrastructure
underpinning CMIP, and improving communication between the teams
overseeing the scientific and experimental design of these globally
coordinated experiments, and the teams providing resources and
designing that infrastructure. The communication was intended to be
two-way: providing input both to the provisioning of infrastructure
appropriate to the experimental design, and informing the scientific
design of the technical (and financial) limits of that infrastructure.

This paper provides a summary of the findings by the WIP in the first
three years of activity since its formation in 2014, and the
consequent recommendations -- in the context of existing
organisational and funding constraints.
\pllabel{RC1-Overview-2}
In the text below, we refer to \emph{findings}, \emph{requirements},
and \emph{recommendations}. Findings refer to observations about the
state of affairs: technologies, resource constraints, and the like,
based upon our analysis. Requirements are design goals that have been
shared with those building the infrastructure, such as the ESGF
software stack. Recommendations are our guidance to the community:
experiment designers, modeling centres, and the users of climate data.

\pllabel{RC1-Overview-1}
The intended audience for the paper is primarily the scientific
community around CMIP6. In particular, we aim to show how the
scientific design of CMIP6 as outlined in \cite{ref:eyringetal2016a}
translates into infrastructural requirements. We hope this will be
instructive to creators of multi-model experiments as to the resource
implications of their experimental design, and for data providers
(modeling centres), explain the sometimes opaque requirements imposed
upon them as a requisite for participation. We believe an explanation
may also be useful who find data acquisition and analysis a technical
challenge, to understand the design of infrastructure in a
resource-constrained environment. Finally, we hope this will be of
interest to general readers of the journal from other geoscience
fields, illuminating the particular character of global data
infrastructure for climate data, where the community of users far
outstrip in numbers and diversity, the Earth system modeling community
itself.

In Section~\ref{sec:principles}, the principles and scientific
rationale underlying the requirements for global data infrastructure
are articulated. In Section~\ref{sec:dreq} the CMIP6 Data Request is
covered: standards and conventions, requirements for modeling centers
to process a complex data request, and projections of data volume. In
Section~\ref{sec:licensing}, recent evolution in how data are archived
is reviewed alongside a licensing strategy consistent with current
practice and scientific principle. In Section~\ref{sec:cite} issues
surrounding data as a citable resource are discussed, including the
technical infrastructure for the creation of citable data, and the
documentation and other standards required to make data a first-class
scientific entity. In Section~\ref{sec:replica} the implications of
data replicas and in Section~\ref{sec:version} issues surrounding data
versioning, retraction, and errata are addressed.
Section~\ref{sec:summary} provides an outlook for the future of global
data infrastructure, looking beyond CMIP6 towards a unified view of
the ``vast machine'' for weather and climate computation and data.


\section{Principles and Constraints}
\label{sec:principles}

This section lays out some of the the principles and constraints which
have resulted from the evolution of infrastructure requirements since
the first CMIP experiment -- beginning with the historical context.

\subsection{Historical Context}
\label{sec:history}

In the pioneering days of CMIP, the community of participants was
small and well-knit, and all the issues involved in generating
datasets for common analysis from different modeling groups could be
settled by mutual agreement (Ron Stouffer, personal communication).
Analysis was performed by the same community that performed the
simulations. The Program for Climate Model Diagnostics and
Intercomparison (PCMDI), established in 1989, had championed the idea
of more systematic analysis of models, and in close cooperation with
the climate modeling centers, PCMDI assumed responsibility for much of
the day-to-day coordination of CMIP. Until CMIP3, the hosting of
datasets from different modeling groups could be managed at a single
archival site; PCMDI alone hosted the entire 40~TB archive.

From its earliest phases, CMIP grew in importance, and its results
provided a major pillar supporting the periodic Intergovernmental
Panel on Climate Change (IPCC) assessment activity. However, the
explosive growth in the scope of CMIP, especially between CMIP3 and
CMIP5, represented a tipping point in the supporting infrastructure.
Not only was it clear that no one site could manage all the data, the
necessary infrastructure software and operational principles could no
longer be delivered and managed by PCMDI alone.

For CMIP5, PCMDI sought help from a number of partners under the
auspices of the Global Organisation of Earth System Science Portals
(GO-ESSP). In the main, the GO-ESSP partners who became the foundation
members and developers of the Earth System Grid Federation retargeted
existing research funding to develop ESGF. The primary heritage was
the original U.S. Earth System Grid Federation project, but major
components came from new international partners. This meant that many
aspects of the ESGF system began from work which was designed in the
context of different requirements, collaborations, and objectives. At
the beginning, none of the partners had funds for operational support
for the fledgling international federation, and even after the end of
CMIP5 proper, the ongoing ESGF has been sustained primarily by small
amounts of funding at a handful of the ESGF sites. Most ESGF sites
have had little or no formal operational support. Many of the known
limitations of the CMIP5 ESGF -- both in terms of functionality and
performance -- were a direct consequence of this heritage.

With the advent of CMIP6, it was clear that
\pllabel{RC2-14}
a fundamental reassessment would be needed to address the evolving
scientific and operational requirements. That clarity led to the
establishment of the WIP, but it has yet to lead to any formal joint
funding arrangement -- the ESGF and the data nodes within it remain
funded (if at all, many data nodes are marginal activities supported
on best efforts) by national agencies with disparate timescales and
objectives. Several critical software elements also are being
developed on volunteer efforts and shoestring budgets. This finding
has been noted in the US National Academies Report on ``A National
Strategy for Advancing Climate Modeling'' \citep{ref:nasem2012}, which
warned of the consequences of inadequate infrastructure funding.

\subsection{Infrastructural Principles}
\label{sec:infra-principles}

\begin{enumerate}
\item With greater complexity and a globally distributed data
  resource, it has become clear that in the design of globally
  coordinated scientific experiments, the global computational and
  data infrastructure needs to be formally examined as an integrated
  element.
  
  The membership of the WIP, drawn as it is from experts in various
  aspects of the infrastructure, is a direct consequence of this
  requirement for integration. Representatives of modeling centers,
  infrastructure developers, and stakeholders in the scientific design
  of CMIP and its output comprise the panel membership. One of the
  WIP's first acts was to consider three phases in the process of
  infrastructure development: \emph{requirements},
  \emph{implementation}, and \emph{operations}, all informed by the
  builders of workflows at the modeling centers.
    
  \begin{itemize}
  \item The WIP, in consort with the CMIP Panel, takes responsibility
    to articulate \emph{requirements} for the infrastructure.
  \item The \emph{implementation} is in the hands of the
    infrastructure developers, principally ESGF for the federated
    archive \citep{ref:williamsetal2015}, but also related projects
    like Earth System Documentation
    \citep[\urlref{https://goo.gl/WNwKD9}{ES-DOC},][]{ref:guilyardietal2013}.
  \item In 2016 at the WIP's request, the CMIP6 Data Node
    \emph{Operations} Team (CDNOT) was formed.
    \pllabel{RC3-22}
    It is charged with ensuring that all the infrastructure elements
    needed by CMIP6 are properly deployed and actually working as
    intended at the sites hosting CMIP6 data. It is also responsible
    for the operational aspects of the federation itself, including
    specifying what versions of the toolchain are run at every site at
    any given time, and organizing coordinated version upgrades across
    the federation.
  \end{itemize}

  Although there is now a clear separation of concerns into
  requirements, implementation, and operations, close links are
  maintained by cross-membership between the key bodies, including the
  WIP itself, the CMIP Panel, the ESGF Executive Committee, and the
  CDNOT.
\item\label{broad} With the basic fact of anthropogenic climate change
  now well established \citep[see, e.g.,][]{ref:stockeretal2013} the
  scientific communities with an interest in CMIP is expanding. For
  example, a substantial body of work has begun to emerge to examine
  climate impacts. In addition to the specialists in Earth system
  science -- who also design and run the experiments and produce the
  model output -- those relying on CMIP output now include those
  developing and providing climate services, as well as
  \emph{consumers} from allied fields studying the impacts of climate
  change on health, agriculture, natural resources, human migration,
  and similar issues \citep{ref:mossetal2010}. This confronts us with
  a \emph{scientific scalability} issue (the data during its lifetime
  will be consumed by a community much larger, both in sheer numbers,
  and also in breadth of interest and perspective than the Earth
  system modeling community itself), which needs to be addressed.

  Accordingly, we note the requirement that infrastructure should
  ensure maximum transparency and usability for user (consumer)
  communities at some distance from the modeling (producer)
  communities.
\item\label{repro} While CMIP and the IPCC are formally independent,
  the CMIP archive is increasingly a reference in formulating climate
  policy. Hence the \emph{scientific reproducibility}
  \citep{ref:collinstabak2014} and the underlying \emph{durability}
  and \emph{provenance} of data have now become matters of central
  importance: being able to trace
  \pllabel{RC2-15}
  back, long after the fact, from model output to the configuration of
  models, and procedures and choices made along the way. This led the
  IPCC to require data distribution centers (DDCs) to attempt to
  guarantee the archival and dissemination of this data in perpetuity,
  and consequently to a requirement in the CMIP context of
  achieving reproducibility. Given the use of multi-model ensembles
  for both consensus estimates and uncertainty bounds on climate
  projections, it is important to document -- as precisely as
  possible, given the independent genealogy and structure of many
  models -- the details and differences among model configurations and
  analysis methods, to deliver both the requisite provenance and the
  routes to reproduction.
\item\label{analysis} With the expectation that CMIP DECK experiment
  results should be routinely contributed to CMIP, opportunities now
  exist for engaging in a more systematic and routine evaluation of
  Earth System Models (ESMs). This has led to community efforts to
  develop standard metrics of model ``quality''
  \citep{ref:eyringetal2016,ref:gleckleretal2016}.
  \pllabel{RC2-16}
  Typical multi-model analysis has hitherto taken the multi-model
  average, assigning equal weight to each model, as the most likely
  estimate of climate response. This ``model democracy''
  \citep{ref:knutti2010} has been called into question and there is
  now a considerable literature exploring the potential of weighting
  models by quality \citep{ref:knuttietal2017}. The development of
  standard metrics would aid this kind of research.

  To that end, there is now a requirement to enable through the ESGF a
  framework for accommodating quasi-operational evaluation tools that
  could routinely execute a series of standardized evaluation tasks.
  This would provide data consumers with an increasingly (over time)
  systematic characterization of models. It may be some time before a
  fully operational system of this kind can be implemented, but
  planning must start now.

  \pllabel{SC1-1}
  In addition, there is an increased interest in climate analytics as
  a service \citep{ref:balajietal2011,ref:schnaseetal2017}. This
  follows the principle of placing analysis close to the data. Some
  centres plan to add resources that combine archival and analysis
  capabilities, e.g., NCAR's \urlref{https://goo.gl/sYTxC2}{CMIP
    Analysis Platform}, or the UK's JASMIN
  \citep{ref:lawrenceetal2013}.. There are also new efforts to bring
  climate data storage and analysis to the cloud era
  \citep[e.g][]{ref:duffyetal2015}. Platforms such as
  \urlref{http://pangeo-data.org/}{Pangeo} show much promise in this
  realm, and widespread experimentation and adoption is encouraged.
\item As the experimental design of CMIP has grown in complexity,
  costs both in time and money have become a matter of great concern,
  particularly for those designing, carrying out, and storing
  simulations. In order to justify commitment of resources to CMIP,
  mechanisms to identify costs and benefits in developing new models,
  performing CMIP simulations, and disseminating the model output need
  to be developed.

  To quantify the scientific impact of CMIP, measures are needed to
  \emph{track} the use of model output and its value to consumers. In
  addition to usage quantification, credit and tracing data usage in
  literature via citation of data is important. Current practice is at
  best citing large data collections provided by a CMIP participant,
  or all of CMIP. Accordingly, we note the need for a mechanism to
  identify and \emph{cite} data provided by each modeling center.
  Alongside the intellectual contribution to model development, which
  can be recognized by citation, there is a material cost to centers
  in computing and data processing, which is both burdensome
  \pllabel{RC1-11}
  and poorly understood by those requesting, designing and using the
  results from
  \pllabel{RC1-12}

  CMIP experiments, who might not be in the business of model
  development. The criteria for endorsement introduced in CMIP6
  \citep[see Table~1 in][]{ref:eyringetal2016a} begins to grapple with
  this issue, but the costs still need to be measured and recorded. To
  begin documenting these costs for CMIP6, the ``Computational
  Performance'' MIP project (CPMIP) \citep{ref:balajietal2017} has
  been established, which will \pllabel{RC1-13} measure, among other
  things, throughput (simulated years per day) and cost (core-hours
  and joules per simulated year) as a function of model resolution and
  complexity. Tools for estimating data volumes have also developed,
  see Section~\ref{sec:data-request} below.
\item\label{cmplx} Experimental specifications have become ever more
  complex, making it difficult to verify that experiment
  configurations conform to those specifications.
  \pllabel{RC2-17}
  Several modeling centers have encountered this problem in preparing
  for CMIP6, noting, for example, the challenging intricacies in
  dealing with input forcing data \citep[see][]{ref:duracketal2017},
  output variable lists \citep{ref:juckesetal2015}, and crossover
  requirements between the endorsed MIPs and the DECK
  \citep{ref:eyringetal2016a} . Moreover, these protocols inevitably
  evolve over time, as errors are discovered or enhancements proposed,
  and centers needed to be adaptable in their workflows accordingly.
   
  We note therefore a requirement to encode the protocols to be
  directly ingested by workflows, in other words,
  \emph{machine-readable experiment design}.
  \pllabel{RC1-14}
  The intent is to avoid, as far as possible, errors in conformance to
  design requirements introduced by the need for humans to transcribe
  and implement the protocols, for instance, deciding what variables
  to save from what experiments. This is accomplished by encoding most
  of the specifications in structured text formats which can be
  directly read by the scripts running the model and post-processing,
  as explained further below in Section~\ref{sec:dreq}. The
  requirement spans all of the \emph{controlled vocabularies} (CVs:
  for instance the names assigned to models, experiments, and output
  variables) used in the CMIP protocols as well as the CMIP6 Data
  Request \citep{ref:juckesetal2015}, which must be stored in
  version-controlled, machine-readable formats. Precisely documenting
  the \emph{conformance} of experiments to the protocols
  \citep{ref:lawrenceetal2012} is an additional requirement.
\item\label{snap} The transition from a unitary archive at PCMDI in
  CMIP3 to a globally federated archive in CMIP5 led to many changes
  in the way users interact with the archive, which impacts management
  of information about users and complicates communications with them.
  In particular, a growing number of data users no longer register or
  interact directly with the ESGF. Rather they rely on secondary
  repositories, often ``snapshots'' of the state of some portion of
  the ESGF archive created by others at a particular time (see for
  instance the \urlref{https://goo.gl/34AtW6}{IPCC CMIP5 Data
    Factsheet}
  \pllabel{RC1-15}
  for a discussion of the snapshots and their coverage). This meant
  that reliance on the ESGF's inventory of registered users for any
  aspect of the infrastructure -- such as tracking usage, compliance
  with licensing requirements, or informing users about errata or
  retractions -- could at best ensure partial coverage of the user
  base.

  This key finding implies a more distributed design for several
  features outlined below, which devolve many of these features to the
  datasets themselves rather than the archives. One may think of this
  as a \emph{dataset-centric rather than system-centric} design (in
  software terms, a \emph{pull} rather than \emph{push} design):
  information is made available upon request at the user/dataset
  level, relieving the ESGF implementation of an impossible burden.
\end{enumerate}

Based upon these considerations, the WIP produced a set of position
papers (see Appendix~\ref{sec:wip}) encapsulating specifications and
recommendations for CMIP6 and beyond. These papers, summarized below,
are available from the
\urlref{https://www.earthsystemcog.org/projects/wip/}{WIP website}. As
the WIP continues to develop additional recommendations, they too will
be made available. As requirements evolve, a modified document will
be released with a new version number.

\section{A structured approach to data production}
\label{sec:dreq}

The CMIP6 data framework has evolved considerably from CMIP5, and
follows the principles of scientific reproducibility (Item~\ref{repro}
in Section~\ref{sec:principles}), and the recognition that the
complexity of the experimental design (Item~\ref{cmplx}) required far
greater degrees of automation and embedding in workflows. This
requires that all elements in the specification be recorded in
structured text formats (XML and JSON, for example), and subject to
rigorous version control. \emph{Machine-readable} specification of as
many aspects of the model output configuration as possible is a
design goal, as noted earlier.

The data request spans several elements discussed in sub-sections
below.

\subsection{CMIP6 Data Request}
\label{sec:data-request}

\pllabel{RC2-18}
The CMIP6 Data Request is one of the most complex elements of the
CMIP6 infrastructure. It is a direct response to the complexity of the
new design outlined in \cite{ref:eyringetal2016a}. The experimental
design now involves 3 tiers of experiments, where an individual
modeling group may choose which ones to perform; and variables grouped
by scientific goals and priorities, where again centres may choose
which sets to publish, based on interests and resource constraints.
There are also cross-experiment data requests, where for instance the
design may require a variable in one experiment to be compared against
the same variable from a different experiment. The modeling groups
will then need to take this into account before beginning their
simulations. The CMIP6 Data Request is a codification of the entire
experimental design into a structured set of machine-readable
documents, which can in principle be directly ingested in data
workflows.

The \urlref{https://goo.gl/iNBQ9m}{CMIP6 Data Request}
\citep{ref:juckesetal2015} combines definitions of variables and their
output format with specifications of the objectives they support and
the experiments that they are required for. The entire request is
encoded in an XML database with rigorous type constraints. Important
elements of the request, such as units, cell methods (expressing the
subgrid processing implicit in the variable definition), and
frequencies and time ``slices'' (subsets of the entire simulation
period as defined in the experimental design) for required output, are
defined as controlled vocabularies within the request to ensure
consistency of usage. The request is designed to enable flexibility,
allowing modeling centers to make informed decisions about the
variables they should submit to the CMIP6 archive from each
experiment.

% The data request spans several elements.

% \begin{enumerate}
% \item specification of the parameter to be calculated in terms of a CF
%   standard name and units,
% \item an output frequency,
% \item a structural specification which includes specification of
%   dimensions and of subgrid processing.
% \end{enumerate}

In order to facilitate the cross linking between the 2100 variables
from 248 experiments, the request database allows MIPs to aggregate
variables and experiments into groups. This allows MIPs to designate
variable groups by priority, and allow queries that return a
\emph{Request}, informing the modeling groups of the variables needed
from any given experiment, at the specified time slices and
frequencies.
% The link between variables and
% experiments is then made through the following chain:

% \begin{itemize}
% \item A \emph{variable group}, aggregating variables with priorities
%   specific to the MIP defining the group;
% \item A \emph{request link} associating a variable group with an
%   objective and a set of request items;
% \item \emph{Request} items associating a particular time slice with a
%   request link and a set of experiments.
% \end{itemize}

This formulation takes into account the complexities that arise when a
particular MIP requests that variables needed for their own
experiments should also be saved from a DECK experiment or from an
experiment proposed by a different MIP.

The data request supports a broad range of users who are 
provided with a range of different access points. These include the
entire codification in the form of structured (XML) document, web
pages, or spreadsheets, as well as a python API and command-line tools
to satisfy a wide variety of usage patterns for accessing and using
the data request.

% \begin{enumerate}
% \item The XML database provides the reference document;
% \item Web pages provide a direct representation of the database
%   content;
% \item Excel workbooks provide selected overviews for specific MIPs and
%   experiments;
% \item A python library provides an interface to the database with some
%   built-in support functions;
% \item A command line tool based on the python library allows quick
%   access to simple queries.
% \end{enumerate}

The data request's machine-readable database has been an extraordinary
resource for the modeling centers. They can, for example, directly
integrate the request specifications with their workflows to ensure
that the correct set of variables are saved for each experiment they
plan to run. In addition, it has given them a new-found ability to
estimate the data volume associated with meeting a MIP's requirements,
a feature exploited below in Section~\ref{sec:dvol}.

\subsection{Model inputs}
\label{sec:data-inputs}

Datasets used by the model for configuration of model inputs
\citep[\texttt{input4MIPs}, see][]{ref:duracketal2017} as well as
observations for comparison with models \citep[\texttt{obs4MIPs},
see][]{ref:teixeiraetal2014} are both now organized in the same way,
and share many of the naming and metadata conventions as the CMIP
model output itself.
\pllabel{RC3-9}
The coherence of standards across model inputs, outputs, and
observational datasets is a development that will enable the community
to build a rich toolset across all of these datasets. The datasets
follow versioning methodologies below in Section~\ref{sec:version}.

\subsection{Data Reference Syntax}
\label{sec:data-drs}

The organization of the model output follows the
\urlref{http://goo.gl/v1drZl}{Data Reference Syntax (DRS)} first used
in CMIP5, and now in somewhat modified form in CMIP6. The DRS depends
on pre-defined \emph{controlled vocabularies} (CVs) for various terms
including: the names of institutions, models, experiments, time
frequencies, etc. The CVs are now recorded as a version-controlled set
of structured text documents, and satisfies the requirement that there
is a \urlref{https://goo.gl/HGafnJ}{single authoritative source for
  any CV}, on which all elements in the toolchain will rely. The DRS
elements that rely on these controlled vocabularies appear as netCDF
attributes and are used in constructing file names, directory names,
and unique identifiers of datasets that are essential throughout the
CMIP6 infrastructure. These aspects are covered in detail in the
\urlref{https://goo.gl/mSe4rf}{CMIP6 Global Attributes, DRS,
  Filenames, Directory Structure, and CVs} position paper. A new
element in the DRS indicates whether data has been stored on a native
grid or has been regridded (see discussion below in
Section~\ref{sec:dvol} on the potentially critical role of regridded
output). This element of the DRS will allow us to track the usage of
the \emph{regridded subset} of data, and assess the relative
popularity of native-grid vs. standard-grid output.

\subsection{CMIP6 data volumes}
\label{sec:dvol}

As noted, extrapolations based on CMIP3 and CMIP5 lead to some
alarming trends in data volume \citep[see
e.g.,][]{ref:overpecketal2011}.
\pllabel{RC3-10}
As seen in their Figure~2, model output such as those from CMIPs are
beginning to rival observational data volume. As noted in the
Introduction, a particular problem for our community is the diverse
and very large user base for the data, many of whom are not climate
specialists, but downstream users of climate data studying the impacts
of climate change. This stands in contrast to other fields with
comparably large data holdings: data from the Large Hadron Collider
\citep[e.g.,][]{ref:aadetal2008} for example, is primarily consumed by
high energy physicists and not of direct interest to scientists in
unrelated fields.

A rigorous approach is needed to the estimation of future
data volumes, rather than simple extrapolation. Contributions to
increase in data volume include the systematic increase in model
resolution and complexity of the experimental protocol and data
request. We consider these separately:

\begin{description}
\item[Resolution] The median horizontal resolution of a CMIP model
  tends to grow with time, and is expected to be more typically 100~km
  in CMIP6, compared to 200~km in CMIP5. The vertical resolution grows
  in a more controlled fashion, at least as far as the data is
  concerned, as often the requested output is reported on a standard
  set of atmospheric levels that has not changed much over the years.
  Similarly the temporal resolution of the data request does not
  increase at the same rate as the model timestep: monthly averages
  remain monthly averages. A doubling of model resolution leads
  therefore to a quadrupling of the data volume, in principle. But
  typically the temporal resolution of the model (though not the data)
  is doubled as well, for reasons of numerical stability. Thus, for an
  $N$-fold increase in horizontal resolution, we require an $N^3$
  increase in computational capacity, which will result in an $N^2$
  increase in data volume. We argue therefore, that data volume $V$
  and computational capacity $C$ are related as $V \sim C^\frac23$,
  purely from the point of view of resolution. The exponent is even
  smaller if vertical resolution increases are assumed.
  \pllabel{RC1-18}
  This is because most 3D model output is requested on sets of
  ``standard levels'' and thus the output fields do not scale with the
  number of model levels (see discussion in the
  \urlref{https://goo.gl/wVtm5t}{CMIP6 Output Grid Guidance
    document}).
  
  If we then assume that centers will experience an 8-fold increase in
  $C$ between CMIPs (which is optimistic in an era of tight budgets),
  we can expect a 4-fold increase in data volume. However, this is not
  what we experienced between CMIP3 and CMIP5. What caused that
  extraordinary 50-fold increase in data volume?
\item[Complexity] The answer lies in the complexity of CMIP: the
  complexity of the data request, and of the experimental protocol.
  The first component, the
  \pllabel{RC1-19}
  data request complexity, is related to that of the science: the
  number of processes being studied, and the physical variables
  required for the study. In CPMIP \citep{ref:balajietal2017}, we have
  attempted a rigorous definition of this complexity, measured by the
  number of physical variables simulated by the model. This, we argue,
  grows not smoothly like resolution, but in very distinct
  generational step transitions, such as the one from atmosphere-ocean
  models to Earth system models, which involved a substantial jump in
  complexity, the number of physical, chemical, and biological species
  being modeled, as shown in \cite{ref:balajietal2017}.
  \pllabel{RC1-29a}
  The dramatic increase in data volume between CMIP3 and CMIP5 was
  also due to these causes. Many models of the CMIP5 era added
  atmospheric chemistry and aerosol-cloud feedbacks, sometimes with
  $\mathcal{O}(100)$ species. CMIP5 also marked the first time in CMIP
  that ESMs were used to simulate changes in the carbon cycle and
  modeling groups performed many more simulations than in CMIP3 with a
  corresponding increase in years simulated.

  % the following increase in complexity doesn't help explain the 50-fold increase 
  % which is what this paragraph is supposed to address
  %  the number of experiments (or number of years simulated) are
  % primarily controlled by $C$, which you say is limited to 8-fold increase.
  %  need to restructure the argument.
  The second component of complexity is the experimental protocol, and 
  the number of experiments themselves when comparing CMIP5 and CMIP6.
  With the new structure of CMIP6, with a DECK and 23 endorsed MIPs,
  this
  \pllabel{RC3-11}
  has grown tremendously. We propose as a measure of experimental
  complexity, the \emph{total number of simulated years (SYs)}
  conforming to a given protocol. Note that this too is gated by $C$:
  modeling centers usually make tradeoffs between experimental
  complexity and resolution in deciding their level of participation
  in CMIP6, discussed in \cite{ref:balajietal2017}.
\end{description}

Two further steps have been proposed toward ensuring sustainable
growth in data volumes.
% Given the earlier arguments, it seems $C$ will limit growth of volume by itself
%  Why are additional steps necessary?
\pllabel{RC2-21}
The first of these is the consideration of standard horizontal
resolutions for saving data, as is already done for vertical and
temporal resolution in the data request. Cross-model analyses already
cast all data to a common grid in order to evaluate it as an ensemble,
typically at fairly low resolution. The studies of Knutti and
colleagues (e.g., \cite{ref:knuttietal2017}) are typically performed
on relatively coarse grids. Accordingly for most purposes
atmospheric data on the ERA-40 grid ($2^\circ\times 2.5^\circ$) would
suffice, with of course exceptions for experiments like those called
for by HighResMIP \citep{ref:haarsmaetal2016}. A similar
conclusion applies for ocean data (the World Ocean Atlas
$1^\circ\times 1^\circ$ grid), with extended discussion of the
benefits and losses due to regridding
\citep[see][]{ref:griffiesetal2014,ref:griffiesetal2016}.
\pllabel{RC3-14}

This has not been mandated for CMIP6 for a number of reasons. Firstly,
regridding is burdensome on many grounds: It requires considerable
expertise to choose appropriate algorithms for particular variables,
for instance, we may need ones that guarantee exact conservation for
scalars or preservation of streamlines for vector fields may be a
requirement; and it can be expensive in terms of computation and
storage. Secondly, regridding is irreversible (thus amounting to
``lossy'' data reduction) and non-commutative with certain basic
arithmetic operations such as multiplication (i.e., the product of
regridded variables does not in general equal the regridded output of
the product computed on the native grid). This can be problematic for
budget studies. However, the same issues would apply for
time-averaging and other operations long used in the field: much
analysis of CMIP output is performed on monthly-averaged data, which
is ``lossy'' compression along the time axis relative to the model's
time resolution.

These issues have contributed to a lack of consensus in moving forward,
and the recommendations on regridding remain in flux. The
\urlref{https://goo.gl/wVtm5t}{CMIP6 Output Grid Guidance document}
outlines a number of possible recommendations, including the provision
of ``weights'' to a target grid. Many of the considerations around
regridding, particularly for ocean data in CMIP6, are discussed at
length in \cite{ref:griffiesetal2016}. 

There is a similar lack of consensus around common \emph{calendar} for
particular experiments.
\pllabel{RC3-13}
In cases such as a long-running control simulation where all years are
equivalent and of no historical significance, it is customary in this
community to use simplified calendars -- such as a Julian, a
``noleap'' (365-day) or ``equal-month'' (360-day) calendar -- rather
than the Gregorian, which can vastly simplify analysis. However,
comparison across datasets using incommensurate calendars can be a
frustrating burden on the end-user. There is no consensus at this
point on this issue.

As outlined below in Section~\ref{sec:replica}, both ESGF data nodes
and the creators of secondary repositories are given considerable
leeway in choosing data subsets for replication, based on their own
interests. The tracking mechanisms outlined in Section~\ref{sec:pid}
below will allow us to ascertain, after the fact, how widely used the
native grid data may be \emph{vis-\`a-vis} the regridded subset, and
allow us to recalibrate the replicas, as usage data becomes available.
We note also that the providers of at least one of the standard
metrics packages \citep[ESMValTool,][]{ref:eyringetal2016a} have
expressed a preference of standard grid data for their analysis, as
regridding from disparate grids increases the complexity of their
already overburdened infrastructure.

A second method of data reduction for the purposes of storage and
transmission, is the issue of data compression. netCDF4, which is the
recommended for CMIP6 data, includes an option for lossless
compression or deflation \citep{ref:zivlempel1977} that relies on the
same technique used in standard tools such as \texttt{gzip}. In
practice, the reduction in data volume will depend upon the
``entropy'' or randomness in the data, with smoother data being
compressed more.

Deflation entails computational costs, not only during creation of the
compressed data, but also every time the data are re-inflated. There
is also a subtle interplay with precision: for instance temperatures
usually seen in climate models appear to deflate better when expressed
in Kelvin, rather than Celsius, but that is due to the fact that the
leading order bits are always the same, and thus the data is actually
less precise. Deflation is also enhanced by reorganizing
(``shuffling'') the data internally into chunks that have spatial and
temporal coherence.

Some in the community argue for the use of more aggressive
\emph{lossy} compression methods \citep{ref:bakeretal2016}, but the
community, after consideration, believes the loss of precision
entailed by such methods, and the consequences for scientific results,
require considerably more evaluation by the community before such
methods can be accepted as common practice. However, as noted above,
some lossy methods of data reduction such as time-averaging, have long
been common practice.

Given the options above, we undertook a systematic study of the
behavior of typical model output files under lossless compression, the
results of which are \urlref{https://goo.gl/qkdDnn}{publicly available}.
The study indicates that standard \texttt{zlib} compression in the
netCDF4 library with the settings of \texttt{deflate=2} (relatively
modest, and computationally inexpensive), and \texttt{shuffle} (which
ensures better spatiotemporal homogeneity) ensures the best compromise
between increased computational cost and reduced data volume. For an
ESM,
\pllabel{RC1-25}
we expect a total savings of about 50\%, with ocean, ice, land realms
getting the most savings (owing to large areas of the globe that are
masked), and atmospheric data the least. This 50\% estimate has been
verified with sample output from some models preparing for CMIP6.

The \urlref{https://goo.gl/iNBQ9m}{DREQ} alluded to above in
Section~\ref{sec:dreq} allows us to make a systematic assessment of
these considerations. The tool expects one to input a model's
resolution along with the experiments that will be performed and the
data one intends to save (using DREQ's \emph{priority} attribute).
With this information
% We are actually capturing this information in the registered content
% for the model source_id entries - see http://rawgit.com/WCRP-CMIP/CMIP6_CVs/master/src/CMIP6_source_id.html
% The json entry contains resolutions for each active model realm
% https://github.com/WCRP-CMIP/CMIP6_CVs/blob/master/CMIP6_source_id.json
%  "unprecedented" is incorrect.
% In CMIP5 we had a sophisticated capability of estimating data volume
%  We polled the groups to determine which experiments they planned
% to run and how large their ensembles would be.  
%  We also asked what resolution they would report output.
%  From this we estimated in Nov. 2010 a total data volume of 2.5 petabytes 
%  (2.1 petabytes if only high-priority variables were reported), not too 
% far from the actual volume.  I'll send you the analysis if you like.
% The modeling groups had access to this information.
\pllabel{RC2-23}
one may calculate the data volume that will be produced. For instance,
analyses available
\urlref{http://clipc-services.ceda.ac.uk/dreq/tab01_3_3.html}{DREQ
  site} indicate that if a center were to undertake every single
experiment (all tiers) and save every single variable requested (all
priorities) at a ``typical'' resolution, it would generate about
800~TB of data, using the guidelines above. Given 75 participating
models, this translates to an upper bound of 60~PB for the entire
CMIP6 archive, though in practice most centers are planning to perform
a subset of experiments, and save a subset of variables, based on
their scientific priorities and available computational and storage
resources. The WIP carried out a survey of modeling centers in 2016,
asking them for their expected model resolutions, and intentions of
participating in various experiments. Based on that survey, we
initially have forecast a
\pllabel{RC1-27}
compressed data volume of 18~PB for CMIP6. This number, 18~PB, is
about 6 times the CMIP
\pllabel{RC1-28}
(including all CMIPs, but the dominant component is CMIP5) archive
size, and can be explained in terms of the compounding of modest
increases in resolution and complexity, as explained above.
\pllabel{RC1-29b}
Causes for a dramatic increase in data volume between CMIP3 and CMIP5
were noted above. There is no comparable jump between CMIP5 and CMIP6.
CMIP6's innovative DECK/endorsed-MIP structure should thus be seen as
an extension and an attempt to impose a rational order on CMIP5,
rather than a qualitative leap.

Similar analyses were undertaken at PCMDI for CMIP5. For CMIP6, tools
have been made available that put this capability in the hands of the
modeling centers themselves.
\pllabel{RC1-26}
In particular, the cross-MIP data requests (variables requested by one
MIP from another MIP, or the DECK) require a sophisticated
understanding of the experimental protocols. The experience in many
modeling centers currently is that data volume estimates are only
available when the production runs have begun. Reliable estimates
\emph{ahead of time} based on nothing more than the experimental
protocols and model resolutions will be of use for planning and
acquisitions.

% if you want to discuss different grids, perhaps here is a better place for
% that.
It should be noted that reporting output on a lower resolution
standard grid (rather than the native model grid) could shrink this
volume 10-fold, to 1.8~PB. This is an important number, as will be
seen below in Section~\ref{sec:replica}: the managers of Tier~1 nodes
\pllabel{RC3-18}
(the largest nodes in the federation) have indicated that 2~PB is
about the practical limit for replicated storage of combined data from
all models.
% I for one don't think it is important for all the data to be replicated
This target is achievable based on compression and the use of standard
grids. Both of these (the use of netCDF4 compression and regridding)
remain merely recommendations, and the centers are free to choose
whether or not to compress and regrid.

\section{Licensing}
\label{sec:licensing}

The licensing policy in force for CMIP6 is based on an examination of
data usage patterns in CMIP5. First, while the licensing policy called
for registration and acceptance of the terms of use, a large fraction,
perhaps a majority of users, actually obtained their data not directly
from ESGF, but from
\pllabel{RC1-33}
third-party copies, such as the ``snapshots'' alluded to above in
Item~\ref{snap}, Section~\ref{sec:principles}. Those users accessing
the data indirectly, as shown in Figure~\ref{fig:dark}, relied on user
groups or their home institutions to make secondary repositories that
could be more conveniently accessed. The WIP
\urlref{https://goo.gl/7vHsPU}{CMIP6 Licensing and Access Control}
position paper refers to the secondary repositories as ``dark'' and
those obtaining CMIP data from those reposistories as ``dark users''
who are invisible to the ESGF system. While this appears to subvert
the licensing and registration policy put in place for CMIP5, this
should not be seen as a ``bootleg'' process: it is in fact the most
efficient use of limited network bandwidth at the user sites.
\pllabel{RC2-29}
The data archive snapshots will host data and offload some of the
network provisioning requirements from ESGF nodes themselves.

We wish, however, to retain the ability for users of these ``dark''
repositories to benefit from the augmented provenance provided by
infrastructure updates, such as being notified of data retractions or
replacements in the case that contributed datasets are found to be
erroneous and replaced.

\begin{figure*}
  \begin{center}
    \includegraphics[width=175mm]{images/WIP-data-process.png}
  \end{center}
  \caption{Typical data access pattern in CMIP5 involved users making
    local copies, and user groups making institutional-scale caches
    from ESGF. Figure courtesy Stephan Kindermann, DKRZ, adapted from
    WIP Licensing White Paper.}
  \label{fig:dark}
\end{figure*}

The proposed licensing policy inverts this and removes the impossible
task of license enforcement from the distribution system, and embraces
the ``dark'' repositories and users. To quote the WIP position paper:

\begin{quote}
  The proposal is that (1) a data license be embedded in the data
  files, making it impossible for users to avoid having a copy of the
  license, and (2) the onus on defending the provisions of the license
  be on the original modeling center...
\end{quote}

\pllabel{RC2-27}
Licenses are now embedded in the files, and all repositories, whether
sanctioned or ``dark'', can be data sources, as seen below in the
discussion of replication (Section~\ref{sec:replica}.
\pllabel{RC2-30}
In the embedded license approach, modeling centers are offered two
choices of \emph{Creative Commons} licenses: data covered by the
\urlref{https://goo.gl/CY5m2v}{Creative Commons Attribution ``Share
  Alike'' 4.0 International License} will be freely available; for
centers with more restrictive policies, the
\urlref{https://goo.gl/KUNUKq}{Creative Commons Attribution
  ``NonCommercial Share Alike'' 4.0 International License}, which
restricts the data to non-commercial use. Further sharing of the data
is allowed, as the license travels with the data. The PCMDI website
provides a link to the current
\urlref{https://pcmdi.llnl.gov/CMIP6/TermsOfUse}{CMIP6 Terms of Use
  webpage}.

\section{Citation and provenance}
\label{sec:cite}

As noted in Section~\ref{sec:principles}, citation requirements flow
from two underlying considerations: one, to provide proper credit and
formal acknowledgment of the authors of datasets; and the other, to
enable rigorous tracking of data provenance and data usage. The
tracking facilitates scientific reproducibility and traceability, as
well as enabling statistical analyses of dataset utility.

In addition to clearly identifying what data have been used in
research studies and who deserves credit for providing that data, it
is essential that the data be examined for quality and that
documentation be made available describing the model and experiment
conditions under which it was generated. These subjects are addressed
in the four position papers summarized in this section.

The principles outlined above are well-aligned with the
\urlref{https://goo.gl/Pzb7F6}{Joint Declaration of Data Citation
Principles} formulated by the Force11 (The Future of Research
Communications and e-Scholarship) Consortium, which has acknowledged
the rapid evolution of digital scholarship and archival, as well as
the need to update the rules of scholarly publication for the digital
age. We are convinced that not only peer-reviewed publications but
also the data itself should now be considered a first-class product of
the research enterprise. This means that data requires curation and
should be treated with the same care as journal articles. Moreover,
most journals and academies now insist that data used in the
literature be made publicly available for independent inquiry and
reproduction of results. New services like
\urlref{http://www.scholix.org}{Scholix} are evolving to support the
exchange and access of such data-data and data-literature
interlinking.

Given the complexity of the CMIP6 data request, we expect, as shown in
Section~\ref{sec:dvol}, a total dataset count of $\mathcal{O}(10^6)$.
Because dozens of datasets are typically used in a single scientific
study, it is impractical to cite each dataset individually in the same
way as individual research publications are acknowledged. The
requirement based on this finding is for a mechanism of citing data
and giving credit to data providers that relies on a rather coarse
granularity, while at the same time offering another option at a much
finer granularity for recording the specific files and datasets used
in a study.

In the following, two distinct types of persistent identifiers (PIDs)
are discussed: DOIs, which can only be assigned to data that comply
with certain standards for citation metadata and curation, and the
more generic
\pllabel{RC1-37}
\urlref{https://www.dona.net/handle-system}{``Handles''} that have
fewer constraints and may be more easily adapted for a particular use.
\pllabel{RC2-31}
The Handle system, as explained below in Section~\ref{sec:pid} allows
unique PIDs to be assigned to datasets at the point of publication.
Technically both types of PIDs rely on the underlying global Handle
System to provide services (e.g., to resolve the PIDs and provide
associated metadata, such as the location of the data itself).

\subsection{Persistent identifiers for acknowledgment and citation}
\label{sec:doi}


Based on earlier phases of CMIP, some datasets initially contributed
to the CMIP6 archive will be flawed (due, for example, to errors in
processing) and therefore will not accurately represent a model's
behavior. When errors are uncovered in the datasets, they may be
replaced with corrected versions. Similarly, additional datasets may
be added to an initially incomplete collection of datasets. Thus,
initially at least, the DOIs assigned for the purposes of citation and
acknowledgement will represent an evolving underlying collection of
datasets.

The recommendations, detailed in the
\urlref{https://goo.gl/BFn9Hq}{CMIP6 Data Citation and Long Term
  Archival} position paper, recognize two phases to the process of
assigning DOI's to collections of datasets: an initial phase, when the
data have been released and preliminary community analysis is still
underway and a second stage when most errors in the data have been
identified and corrected. Upon reaching stage two, the data will be
transferred to long-term archival (LTA) of the IPCC Data Distribution
Centre (IPCC DDC) and deemed appropriate for interdisciplinary use
(e.g., in policy studies). The timing of the planned DDC snapshot is
linked to the IPCC AR6 schedule.

For evolving dataset aggregations, the data citation infrastructure
relies on information collected from the data providers and uses the
\urlref{https://www.datacite.org/dois.html}{DataCite} data
infrastructure to assign DOIs and record associated metadata.
DataCite is a leading global non-profit organisation that provides
persistent identifiers (DOIs) for research data. The DOIs will be
assigned to:

\begin{enumerate}
\item aggregations that include all the datasets contributed by one
  model from one institution from all of a single MIP's experiments,
  and
\item aggregations that include all datasets contributed by one model
  from one institution generated in performing one experiment (which
  might include one or more simulations).
\end{enumerate}

These aggregations are dynamic as far as the PID infrastructure is
concerned: new elements can be added to the aggregation without
modifying the PID. As an example, for the coarser of the two
aggregations defined above, the same PID will apply to an evolving
number of simulations as new experiments are performed with the model.
This PID architecture is shown in Figure~\ref{fig:pidarch}. Since these
collections are dynamic, citation requires authors to provide a
version reference.

\begin{figure*}
  \begin{center}
    \includegraphics[width=175mm]{images/PID-architecture.png}
  \end{center}
  \caption{Schematic PID architecture, showing layers in the PID
    hierarchy. In the lower layers of the hierarchy, PIDs are static
    once generated, and new datasets generate new versions with new
    PIDs. Each file carries a PID and each collection (dataset,
    simulation, ..) is related to a PID. Resolving the PID in the
    Handle server guides the user to the file or the landing page
    describing the collection. Each box in the figure will be
    addressed uniquely by its PID.}
  \label{fig:pidarch}
\end{figure*}

For the stable dataset collections, the data citation infrastructure
requires some additional steps to meet formal requirements. First, we
ensure that there has been sufficient community examination of the
data
\pllabel{RC1-40}
(through citations in published literature, for instance) to qualify
it as having been peer-reviewed. Second, further steps are undertaken
to assure important information exists in ancillary metadata
repositories, including, for example, documentation (ES-DOC, errata
and citation) and to provide quality assurance of data and metadata
consistency and completeness (see Section~\ref{sec:qa}). Once these
criteria have been satisfied, a DOI will be issued by the IPCC DDC
hosted by DKRZ. These dataset collections will meet the stringent
metadata and documentation requirements of the IPCC DDC. Since these
collections are static, no version reference is required in a
citation.
\pllabel{RC1-41}
Should errors be found subsequently, they will be corrected in the
data and published under a new DOI. The original DOI and its related
data are still available but are signed as superseded with link to the
corrected data.

For CMIP6, the initially assigned DOIs (associated with evolving
collections of data) must be used in research papers to properly give
credit to each of the modeling groups providing the data. Once a
stable collection of datasets has met the higher standards for
long-term curation and quality, the DOI assigned by the IPCC DDC
should be used instead.

The data citation approach is described in greater detail in \cite{ref:stockhauselautenschlager2017}.

\subsection{Persistent identifiers for tracking, provenance, and
  curation}
\label{sec:pid}

Although the DOIs assigned to relatively large aggregations of
datasets are well suited for citation and acknowledgment purposes,
they are not issued at fine enough granularity to meet the scientific
imperative that published results should be traceable and verifiable.
Furthermore, management of the CMIP6 archive requires that PIDs be
assigned at a much finer granularity than the DOIs. For these
purposes, PIDs recognized by the global Handle registry will be
assigned at two different levels of granularity:

A unique Handle will be generated each time a new CMIP6 data file is
created, and the Handle will be recorded in the file's metadata (in
the form of a netCDF global attribute named \texttt{tracking\_id}). At
the time the data is published, the \texttt{tracking\_id} will be
processed by the CMIP6 Handle service infrastructure and recorded in
the ESGF metadata catalog. Another Handle will subsequently be
assigned at somewhat coarser granularity to each aggregation of files
containing the data from a single variable sampled at a single
frequency
\pllabel{RC1-45}
from a single model running a single experiment. In ESGF terminology,
this collection of files is referred to as an \emph{atomic dataset}.

As described in the \urlref{https://goo.gl/miUREw}{CMIP6 Persistent
  Identifiers Implementation Plan} position paper, a Handle assigned
at either of these two levels of the PID hierarchy identifies a static
entity; if any file associated with a Handle is altered in any way a
new Handle must be created. The PID infrastructure is also central to
the replication and versioning strategies, as described in
Section~\ref{sec:replica} and Section~\ref{sec:version} below.
Furthermore, as a means of recording provenance and enabling tracking
of dataset usage, authors are urged to include as supplementary
material attached to each CMIP6-based publication a PID list (a flat
list of all PIDs referenced).

\begin{figure*}
  \begin{center}
    \includegraphics[width=175mm]{images/PID-workflow.png}
  \end{center}
  \caption{PID workflow, showing the generation and registry of PIDs,
    with checkpoints where compliance is assured.}
  \label{fig:pidflow}
\end{figure*}

The implementation plan describes methods for generating and
registering Handles using an asynchronous messaging system known as
RabbitMQ. This system, designed in collaboration with ESGF developers
and shown in Figure~\ref{fig:pidflow}, guarantees, for example, that
PIDs are correctly generated in accordance with the versioning
guidelines. The CMIP6 handle system builds on the idea of tracking-ids
used in CMIP5, but with a more rigorous quality control to ensure that
new PIDs are generated when data are modified. The dataset and file
Handles are also associated with basic metadata, called PID Kernel
information \citep{ref:zhouetal2018}, which facilitate the recording
of basic provenance information. Datasets and files point to each
other to bind the granularities together. In addition, dataset kernel
information refers to previous and later versions, errata information
and replicas, explained in more detail in the position paper.

\subsection{Quality Assurance}
\label{sec:qa}

Quality assurance (QA) encompasses the entire data lifecycle, as
depicted in Figure~\ref{fig:qa}. At all stages, a goal is to capture
provenance information that will enable scientific reproducibility.
Further, as noted in Item~\ref{broad} in Section~\ref{sec:principles},
the QA procedures should uncover issues that might undermine trust in
the data by those outside the Earth system modeling community if
errors were left unreported.

\begin{figure*}
  \begin{center}
    \includegraphics[width=175mm]{images/WIP-QA.png}
  \end{center}
  \caption{Schematic of the phases of quality assurance, with earlier
    stages in the hands of modeling centers (left), and more formal
    long-term data curation stages at right. Quality assurance is
    applied both to the data (D, above) as well as the metadata (M)
    describing the data. Figure drawn from the WIP's Quality Assurance
    position paper.}
  \label{fig:qa}
\end{figure*}

QA must ensure that the data and metadata correctly reflect a model's
simulation, so that it can be reliably used for scientific purposes.
As depicted in Figure~\ref{fig:qa}, the first stage of QA is the
responsibility of the data producer: in fact the cycle of model
development and diagnosis is the most critical element of QA. The
second aspect is ensuring that disseminated data include common
metadata based on common CVs, which will enable consistent treatment
of data from different groups and institutions. These requirements are
directly embedded in the ESGF publishing process and in tools such as
\urlref{https://cmor.llnl.gov/}{CMOR} (and its validation component,
\urlref{https://goo.gl/ApvMJx}{PrePARE}). These checks (the D1 and M1
phases of QA in Figure~\ref{fig:qa}) ensure that the data conform to
the CMIP6 Data Request specifications, conform to all naming
conventions and CVs, and follow the mandated structure for
organization into a common directory structure. As noted in
Section~\ref{sec:dreq}, many modeling centers have chosen to embed
these steps directly in their workflows to ensure conformance with the
CMIP6 as the models are being run and their output processed.

At this point, as noted in Figure~\ref{fig:qa}, control is ceded to
the ESGF system, where designated QA nodes
\pllabel{RC1-51}
(ESGF data nodes where additional services are turned on, to certify
for the data for citation and long-term archival) perform further QA
checks. A critical step is the assignment of PIDs
(Section~\ref{sec:pid}, the D2 stage of Figure~\ref{fig:pidflow}),
which is more controlled than in CMIP5 and guarantees that across the
data lifecycle, the PIDs will be reliably useful as unique labels of
datasets.

Beyond this, further stages of QA will be handled within the ESGF
system following procedures outlined in the
\urlref{https://goo.gl/eEr8bS}{CMIP6 Quality Assurance} position
paper. As described before, once data have been published, the data
will be scrutinized by researchers in what can be considered an
ongoing period of community-wide scientific QA of the data. During
this period, modeling centers may correct errors and provide new
versions of datasets. In the final stage, the data pass into long term
archival (LTA) status, described as the ``bibliometric'' phase in
Figure~\ref{fig:qa}. Just prior to LTA, the system will verify minimum
standards of provenance documentation. This is described in the next
section.

\subsection{Documentation of provenance}
\label{sec:doc}

As noted earlier in Section~\ref{sec:dreq}, for data to become a
first-class scientific resource, the methods of their production must
be documented to the fullest extent possible. For CMIP6, this includes
documenting both the models and the experiments. While traditionally
this is done through peer-reviewed literature, which remains
essential, we note that to facilitate various aspects of search,
discovery and tracking of datasets, there is an additional need for
structured documentation in machine readable form.

\begin{figure*}
  \begin{center}
    \includegraphics[width=120mm]{images/cmip6_workflow_infographic_v2.pdf}
  \end{center}
  \caption{Elements of ES-DOC documentation. Rows indicate phases of
    the modeling process being documented, and box colors indicate the
    parties responsible for producing the documentation (see legend).
    Figure courtesy Guillaume Levavasseur, IPSL}.
  \label{fig:esdoc}
\end{figure*}

In CMIP6, the documentation of \emph{experiments}, \emph{models} and
\emph{simulations} is done through the Earth System Documentation
\citep[\urlref{https://goo.gl/WNwKD9}{ES-DOC},][]{ref:guilyardietal2013}
Project. The various aspects of model documentation are shown in
Figure~\ref{fig:esdoc}, and in greater detail in the WIP position
paper on \urlref{https://goo.gl/S3vVxE}{ES-DOC}. The CMIP6
experimental design has been translated into structured text
documents, already available from ES-DOC. ES-DOC has constructed CVs
for the description of the CMIP6 standard model realms
\pllabel{RC1-52}
(CMIP terminology for climate subsystems, such as ``ocean'' or
``atmosphere''), including a set of short tables
(\emph{specialisations}, in ES-DOC terminology) for each realm.
\pllabel{RC1-53}
The specialisations are a succinct and structured description of the
model physics. Ideally, modeling groups would integrate with their
model development process their provision of documentation to ES-DOC.
This will better ensure the accuracy and consistency of the
documentation. ES-DOC provides a variety of user interfaces to read
and write structured documentation that conforms with the Common
Information Model (CIM) of \cite{ref:lawrenceetal2012}. As models
evolve or differentiate (for example, an Earth system model derived
from a particular general circulation model), branches and new
versions of the documentation can be produced
\pllabel{RC1-54}
and it will be possible to display, annotate, and add new entries in
the genealogy of a model in a manner familiar to anyone who works with
version control software like \texttt{git}.

A critical element in the ES-DOC process is the documentation of
\emph{conformances}: steps undertaken by the modeling centers to
ensure that the simulation was conducted as called for by the
experiment design. It is here that we rigorously document which input
datasets were used in a simulation \citep[e.g., the version of each of
the forcing datasets, see][]{ref:duracketal2017}. The conformances
will be an important element in guiding selection of subsets of CMIP6
model results for particular research studies. A researcher might, for
example, choose to subselect only those models that used a particular
version of the forcing datasets that are imposed as part of the
experimental protocol. The conformances will continue to grow in
importance under the CMIP vision that the DECK will provide an ongoing
foundation on which to build a series of future CMIP phases
\citep[shown schematically in Figure~1 of][]{ref:eyringetal2016a}. The
conformances will be essential in enabling studies across model
generations.

The method of capturing the conformance documentation is a two-stage
process that has been designed to minimize the amount of work required
by a modeling center. The first stage is to capture the many
conformances common to all simulations. ES-DOC will then automatically
copy these common conformances to multiple simulations thereby
eliminating duplicated effort. This is followed by a second stage in
which those conformances that are specific to individual experiments
or simulations are collected.

While this method of documentation is unfamiliar to many, such methods
are likely to become common and required practice in the maturing
digital age as part of best scientific practices. Documentation of
software validation \citep[see e.g][]{ref:peng2011} and structured
documentation of complete scientific workflows that can be
independently read and processed are both becoming more common
\citep[see the special issue on the ``Geoscience Paper of the
Future'', ][]{ref:davidetal2016}. We have noted earlier
\pllabel{RC2-32}
(see Item~\ref{repro} in Section~\ref{sec:principles}) the special
importance in climate research today of documenting how results have
been obtained and enabling results to be reproduced by others.
Rigorous documentation remains a hardy bulwark against challenges to
the scientific process.

In keeping with the ``dataset-centric rather than system-centric''
approach (Item~\ref{snap} in Section~\ref{sec:principles}), a user
will be directly linked to documentation from each dataset. This is
done in CMIP6 by
\pllabel{RC1-55}
adding a required global attribute \texttt{further\_info\_url} in file
headers pointing to the associated CIM document, which will serve as
the landing page for documentation from which further exploration (by
humans or software) will take place. The form of this URL is standard
and can be software-generated: CMOR, for instance, will automatically
add it. The existence and functioning of the landing page is assured
in Stage~M3 of Figure~\ref{fig:qa}.

\section{Replication}
\label{sec:replica}

\pllabel{RC3-21}
The replication strategy is covered in the
\urlref{https://goo.gl/Bs4Qou}{CMIP6 Replication and Versioning}
position paper. The recommendations therein are based on the following
\emph{primary} goal:

\begin{itemize}
\item Ensuring at least one copy of a dataset is present at a stable
  ESGF node with a mission of long-term maintenance and curation of
  data. The total data storage resources planned across the Tier~1
  nodes in the CMIP6 era is adequate to support this requirement,
  though some data will likely be held on accessible tape storage
  rather than spinning disk.
\end{itemize}

In addition, we have articulated a number of secondary goals:

\begin{itemize}
\item Enhancing data accessibility across the ESGF (e.g. Australian
  data easily accessible to the European continent despite the long
  distance);
\item Enabling each Tier 1 data node to enact specific policies to
  support their local objectives;
\item Ensuring that the most widely requested data is the most
  accessible across the ESGF federation
  \pllabel{RC1-58}
  (of course, any dataset will be available at least on its original
  publication datanode);
\item Enabling large-scale data analysis across the federation (see
  Item~\ref{analysis} in Section~\ref{sec:principles});
\item Ensuring continuity of data access in the event of individual
  node failures;
\item Enabling network load-balancing and enhanced performance;
\item Reducing the manual workload related to replication;
\item Building a reliable replication mechanism that can be used not
  only within the federation, but by the secondary repositories
  created by user groups (see discussion in
  Section~\ref{sec:licensing} around Figure~\ref{fig:dark}).
\end{itemize}

In conjunction with the ESGF and the International Climate Networking
Working Group (ICNWG), these recommendations have been translated to a
two-pronged strategy.

The basic toolchain for replication is built on updated versions of
the software layers used in CMIP5 including:
\urlref{https://github.com/Prodiguer/synda}{synda} (formerly
\texttt{synchrodata}) and Globus Online \citep{ref:chardetal2015}, which
are based on underlying data transport mechanisms such as
\urlref{https://goo.gl/Z8xcfE}{gridftp} and the older and now deprecated
protocols like \texttt{wget} and \texttt{ftp}.

As before, these layers can be used for \emph{ad hoc} replication by
sites or user groups. For \emph{ad hoc} replication, there is no
obvious mechanism for triggering updates or replication when new data
are published (or retracted, see Section~\ref{sec:version} below).
Therefore, designated \emph{replica nodes} will maintain a protocol
for automatic replication, shown in Figure~\ref{fig:replica}.

\begin{figure*}
  \begin{center}
    \includegraphics[width=120mm]{images/WIP-replication.png}
  \end{center}
  \caption{CMIP6 replication from data nodes to replica centers and
    between replica centers coordinated by a CMIP6 replication team,
    under the guidance of the CDNOT.}
  \label{fig:replica}
\end{figure*}

Given the nature of some of the secondary goals listed above, it would
not be appropriate to prescribe which data should be replicated by
each center. Rather, the plan should be flexible to accommodate
changing data use profiles and resource availability.
\pllabel{RC1-61}
A replication team under the guidance of the CDNOT will coordinate the
replication activities of the CMIP6 data nodes such that the primary
goal is achieved and an effective compromise for the secondary goals
is established.

The International Climate Network Working Group (ICNWG), formed under
the Earth System Grid Federation (ESGF), helps set up and optimize
network infrastructures for ESGF climate data sites located around the
world. For example prioritising the most widely requested data for
replication can best be done based on operational experience and will
of course change over time. To ensure that the replication strategy is
responding to user need and data node capabilities, the replication
team will maintain and run a set of monitoring and notification tools
assuring that replicas are up-to-date. The CDNOT is tasked with
ensuring the deployment and smooth functioning of replica nodes.

A key issue that emerged from discussions with node managers is that
the replication target has to be of sustainable size. A key finding is
that a replication target about 2~PB in size is the practical
(technical and financial) limit for CMIP6 online (disk) storage at any
single location. Replication beyond this may involve offline storage
(tape) for disaster recovery.

Based on experience in CMIP5, it is expected that a number of
``special interest'' secondary repositories will hold selected subsets
of CMIP6 data outside of the ESGF federation. This will have the
effect of widening data accessibility geographically, and by user
communities, with obvious benefit to the CMIP6 program. These
secondary repositories will be encouraged and supported where it does
not undermine CMIP6 data management and integrity objectives.

\pllabel{RC1-62}
In the new dataset-centric approach, licenses and PIDs remain embedded
and will continue to play their roles in the data toolchain even for
these secondary repositories.

In CMIP5 a significant issue for users of some third-party archives
was that their replicated data was taken as a one-time snapshot (see
discussion above in Item~\ref{snap} in Section~\ref{sec:principles}),
and not updated as new versions of the data were submitted to the
source ESGF node. Tools have been developed by a number of
organisations to maintain locally synchronized archives of CMIP5 data
and third party providers should be encouraged to make use of these
types of tools to keep the local archives up to date.

In summary, the requirements for replication are limited to ensuring:

\begin{itemize}
\item that there is at least one instance of each submitted dataset
  stored at a Tier~1 node (in addition to its primary residence)
  within a reasonably short time period following submission;
\item that subsequent versions of submitted datasets are also
  replicated by at least one Tier~1 node (see versioning discussion
  below in Section~\ref{sec:version});
\item that creators of secondary repositories take advantage of the
  replication toolchain described here, to maintain replicas that can
  be kept up to date, rather than one-time snapshots
\item that the CDNOT is the recognized body to manage the operational
  replication strategy for CMIP6.
\end{itemize}

\pllabel{RC1-57}
We note that the the ESGF PID registration service is part of the ESGF
data publication implementation, and not exclusive to CMIP6. The PID
registration service works for all NetCDF-CF files which carry a PID
as \texttt{tracking\_id} field. This is agreed for all CMIP6 data
files. However, the ESGF PID registration service is not exclusively
applicable for CMIP6 model data file but can also be used for new data
sets, subsets, averages and so forth providing the data are in
NetCDF-CF format with a PID from the Handle service in the
\texttt{tracking\_id}. Once the data files passed the ESGF PID
registration service these files may easily be matter of overvalue
services like building collections in the PID hierarchy as given in
Figure~\ref{fig:pidarch}. In general all files as digital objects can
be assigned a PID and registered in the CNRI Handle server. Vice
versa, these objects (files) can be uniquely resolved by the Handle
server providing the PID is known. That means the PID service allows
for stable and transparent data access independently from the actual
storage location. The storage location is part of the PID meta data
which are integrated in the in the Handle server. The PID metadata
generation and registration is part of the ESGF registration service
for NetCDF-CF files but in general the PID architecture is not
restricted to them. It is open for all digital objetcs.

Thus, CMIP6 is the first implementation of the PID service in a larger
data project and ESGF provides in parallel the classical data access
via the Data Reference Syntax outlined in the
\urlref{https://goo.gl/mSe4rf}{CMIP6 Global Attributes, DRS,
  Filenames, Directory Structure, and CVs} position paper.

\section{Versioning}
\label{sec:version}

Versioning strategy (see the \urlref{https://goo.gl/Bs4Qou}{CMIP6
  Replication and Versioning} position paper) is based on the
principle (Section~\ref{sec:principles}) of scientific
reproducibility. Recognizing that errors may be found after datasets
have been distributed, erroneous datasets that may have been used
downstream will continue to be publicly available, but marked as
superseded. This will allow users to trace the provenance of published
results, even if those point to retracted data; and further allow the
possibility of \emph{a~posteriori} correction of such results.

A consistent versioning methodology across all the ESGF data nodes is
required to satisfy these objectives. We note that inconsistent or
informal versioning practices at individual nodes would likely be
invisible to the ESGF infrastructure (e.g., yielding files that look
like replicas, but with inconsistent data and checksums), which would
inhibit traceability across versions.

Building on the replication strategy, and in close consultation with
the ESGF implementation teams, versioning will leverage the PID
infrastructure of Section~\ref{sec:cite}. PIDs are permanently
associated with a dataset, and new versions will get a new PID. When
new versions are published, there will be two-way links created within
the PID kernel information so that one may query a PID for prior or
subsequent versions.

The unit of versioning be an \emph{atomic dataset}: a complete
timeseries of one variable from one experiment and one model. The
implication is that other variables need not be republished, if the
error is found in a single variable. If an entire experiment is
retracted and republished, all variables will get a consistent version
number. The CDNOT will ensure consistent versioning practices at all
participating data nodes.

\subsection{Errata}
\label{sec:errata}

% The following description of CMIP5 errata is not quite right and should
% be revised.
It is worth highlighting in particular the new recommendations
regarding errata. Until CMIP5, we have relied on the ESGF system to
push notifications to registered users regarding retractions and
reported errors. This was found to result in imperfect coverage: as
noted in Section~\ref{sec:licensing}, a substantial fraction of users
are invisible to the ESGF system. Therefore, following the discussion
in Section~\ref{sec:principles} (see Item~\ref{snap}), we have
recommended a design which is dataset-centric rather than
system-centric. Notifications are no longer pushed to users; rather
they will be able to query the status of a dataset they are working
with. An \emph{errata client} will allow the user to enter a PID to
query its status; and an \emph{errata server} will return the PIDs
associated with prior or posterior versions of that dataset, if any.
Details are to be found in the \urlref{https://goo.gl/fvVTVo}{Errata}
position paper.

\conclusions[The future of the global data infrastructure]
\label{sec:summary}

The WIP was formed in response to the explosive growth of CMIP between
CMIP3 and CMIP5, and charged with studying and making recommendations
about the global data infrastructure needed to support CMIP6 and the
future evolution of intercomparison projects. Our findings reflect the
fact that CMIP is no longer a cottage industry, and a more formal
approach is needed. Several of the findings have been translated into
requirements on the design of the underlying software infrastructure
for data production and distribution. We have separated infrastructure
development into requirements, implementation and operations phases,
and provided recommendations on the most efficient use of scarce
resources. The resulting recommendations stop well short of any sort
of global governance of this ``vast machine'', but list many areas
where, with a relatively light touch, beneficial order and control,
and resource efficiencies result.

One key finding that informs everything is that that universally, it
appears that the critical importance of such infrastructure is
under-appreciated. Building infrastructure using research funds puts
the system in an untenable position, with a fundamental contradiction
at its heart: infrastructure by its nature should be reliable, robust,
based on what is proven to work, and invisible, whereas scientific
research is hypothesis-driven, risky and novel, and its details widely
broadcast. While recommendations have been made at the highest level
\citep[e.g.,][]{ref:nasem2012}, there is little progress on this front
to report. Several of the key pieces of infrastructure software
described here are built and tested by volunteers, or short-term
project staff.

The central theme of this paper is the inversion of the design of
federated data distribution, to make it \emph{dataset-centric rather
  than system-centric}. We believe that this one aspect of the design
considerably reduces systemic risk, and allows the size of the system
to scale up and down as resource constraints allow. Individual
scientists, or institutions, or consortia, will be able to pool
resources and share data at will, with relatively light requirements
related to licensing (Section~\ref{sec:licensing}) and dataset
tracking (Section~\ref{sec:pid}). This relieves a considerable design
burden from the ESGF software stack, and further, recognizes that the
data ecosystem extends well beyond the reach of any software system
and that data will be used and reused in myriad ways outside anyone's
control.

A second key element of the design is the insistence on
\emph{machine-readable experimental protocols}. Standards,
conventions, and vocabularies are now stored in machine-readable
structured text formats like XML and JSON, thereby enabling software
to automate aspects of the process. This meets an existing urgent
need, with some modeling centers already exploiting this structured
information to mitigate against the overwhelming complexity of
experimental protocols. Moreover, this will also enable and encourage
unanticipated future use of the information in developing new software
tools for exploiting it as technologies evolve. Our ability to predict
(whether correctly or not remains to be seen) the expected CMIP6 data
volume is one such unexpected outcome.

Finally, the infrastructure allows user communities to assess the
\emph{costs of participation} as well as the benefits. For example, we
believe the new PID-based methods of dataset tracking will allow
centers to measure which data has value downstream. The importance of
citations and fair credit for data providers is recognized, with a
design that facilitates and encourages proper citation practices.
Tools have been added and made available that allow centers, and the
CMIP itself, to estimate data requirements of each experimental
protocol. Ancillary activities such as CPMIP add to this an accounting
of the computational burden of CMIP6.

Certainly not all issues are resolved, and the validation of some of
our findings will have to await the outcome of CMIP6. There is no
community consensus on some proposed design elements, such as standard
grids. Some features long promised, such as server-side analytics
(``bringing analysis to the data'') are yet to mature, although many
exciting efforts are underway, for instance using cloud technologies.
Nevertheless, the discussion in this article provides a sound basis
for beginning to think about the future.

The future brings with it new challenges. First among these is an
expansion of the data ecosystem. There is an increasing blurring of
the boundary between weather and climate as time and space scales
merge \citep{ref:hoskins2013}. This will increasingly entrain new
communities into \pllabel{RC1-64} climate data ecosystems, each with
their own modeling and analysis practices, standards and conventions,
and other issues. The establishment of the WIP was a crucial step in
enhancing the capabilities, standards, protocols and policies around
the CMIP enterprise. Earlier discussions on the scope of the WIP also
suggested a broader scope for the panel on the longer-term, to
coordinate not only the CMIP data aspects (including for example, the
CORDEX project \citep{ref:lakeetal2017}, which also relies upon ESGF
for data dissemination) but also the climate prediction (seasonal to
decadal) issues and corresponding observational and reanalysis
aspects.We would recommend a closer engagement between these
communities in planning the future of a seamless global data
infrastructure.

A further challenge the WIP, and the community must grapple with is
the evolution of scientific publication in the digital age, beyond the
peer-reviewed paper. We have noted above that the nature of
publication is changing \citep[see e.g][]{ref:davidetal2016}. Journals
and academies increasingly insist upon transparency with respect to
code and data to ensure reproducibility. In the future, datasets and
software with provenance information will be first-class entities of
scientific publication, alongside the traditional peer-reviewed
article. In fact it is likely that those will increasingly feature in
the grey literature and scientific social media: one can imagine blog
posts and direct annotations on the published literature around CMIP6
using analysis directly performed on datasets using their PIDs. Data
analytics at large scale is increasingly moving toward machine
learning and other directly data-driven methods of analysis, which
will also be dependent on data labeled with machine-readable metadata.
Our community needs to pay increasing heed to the status of their
data, metadata, and software in the light of these developments.

Future development of the WIP's activities beyond the delivery of
CMIP6 will include an analysis of how the infrastructure design
performed during CMIP6. That analysis, combined with our assessment of
technological change and emerging novel applications, will inform
future design of infrastructure software, as well as recommendations
to the designers of experiments on how best to fit their protocols
within resource limitations. The vision, as always, is for an open
infrastructure that is reliable and invisible, and allows Earth system
scientists to be nimble in the design of collaborative experiments,
creative in their analysis, and rapid in the delivery of the results.

\appendix

\section{List of WIP position papers}
\label{sec:wip}


\begin{itemize}
\item \urlref{https://goo.gl/4A1Xtq}{CDNOT Terms of Reference}: a
  charter for the CMIP6 Data Node Operations Team. Authorship: WIP.
\item \urlref{https://goo.gl/mSe4rf}{CMIP6 Global Attributes, DRS,
    Filenames, Directory Structure, and CVs}: conventions and
  controlled vocabularies for consistent naming of files and
  variables. Authorship: Karl E. Taylor, Martin Juckes, V. Balaji,
  Luca Cinquini, Sébastien Denvil, Paul J. Durack, Mark Elkington,
  Eric Guilyardi, Slava Kharin, Michael Lautenschlager, Bryan
  Lawrence, Denis Nadeau, and Martina Stockhause, and the WIP.
\item \urlref{https://goo.gl/miUREw}{CMIP6 Persistent Identifiers
    Implementation Plan}: a system of identifying and citing datasets
  used in studies, at a fine grain. Authorship: Tobias Weigel, Michael
  Lautenschlager, Martin Juckes and the WIP.
\item \urlref{https://goo.gl/Bs4Qou}{CMIP6 Replication and Versioning}:
  a system for ensuring reliable and verifiable replication; tracking
  of dataset versions, retractions and errata. Authors: Stephan
  Kindermann, Sebastien Denvil and the WIP.
\item \urlref{https://goo.gl/eEr8bS}{CMIP6 Quality Assurance}: systems
  for ensuring data compliance with rules and conventions listed
  above. Authorship: Frank Toussaint, Martina Stockhause, Michael
  Lautenschlager and the WIP.
\item \urlref{https://goo.gl/BFn9Hq}{CMIP6 Data Citation and Long Term
    Archival}: a system for generating Document Object Identifies
  (DOIs) to ensure long-term data curation. Authorship: Martina
  Stockhause, Frank Toussaint, Michael Lautenschlager, Bryan Lawrence
  and the WIP.
\item \urlref{https://goo.gl/7vHsPU}{CMIP6 Licensing and Access
    Control}: terms of use and licenses to use data. Authorship: Bryan
  Lawrence and the WIP.
\item \urlref{https://goo.gl/jWfrWb}{CMIP6 ESGF Publication
    Requirements}: linking WIP specifications to the ESGF software
  stack, conventions that software developers can build against.
  Authorship: Martin Juckes and the WIP.
\item \urlref{https://goo.gl/fvVTVo}{Errata System for CMIP6}: a system
  for tracking and discovery of reported errata in the CMIP6 system.
  Authorship: Guillaume Levavasseur, Sébastien Denvil, Atef Ben
  Nasser, and the WIP.
\item \urlref{https://goo.gl/S3vVxE}{ESDOC Documentation}: An overview
  of the process for providing structured documentation of the models,
  experiments and simulations that produce the CMIP6 output datasets,
  by the ES-DOC Team.
\end{itemize}

\section{Data and code availability}
\label{sec:code}

\begin{itemize}
\item The software and data used for the study of data compression are
  available at \url{https://goo.gl/qkdDnn}, courtesy Garrett Wright.
\item The software and data used for the prediction of data volumes
  are available at \url{https://goo.gl/Ezz5v3}, courtesy Nalanda
  Sharadjaya. Much of this functionality has now been absorbed into
  DREQ itself.
\end{itemize}

Most of the software referenced here for which the WIP is providing
design guidelines and requirements, but not implementation, including
the ESGF, ESDOC, DREQ software stacks are open source and freely
available. They are autonomous projects and therefore not listed here.

\begin{acknowledgements}
  We thank Michel Rixen, Stephen Griffies, and John Krasting for their
  close reading and comments on early drafts of this manuscript.
  Colleen McHugh aided with the analysis of data volumes.
  
  The research leading to these results has received funding from the
  European Union Seventh Framework program under the IS-ENES2 project
  (grant agreement No. 312979).

  V. Balaji is supported by the Cooperative Institute for Climate
  Science, Princeton University, Award NA08OAR4320752 from the
  National Oceanic and Atmospheric Administration, U.S. Department of
  Commerce. The statements, findings, conclusions, and recommendations
  are those of the authors and do not necessarily reflect the views of
  Princeton University, the National Oceanic and Atmospheric
  Administration, or the U.S. Department of Commerce.

  B.N. Lawrence acknowledges additional support from the UK Natural
  Environment Research Council.
  
  K.E. Taylor and P.J. Durack are supported by the Regional and Global
  Model Analysis Program of the United States Department of Energy's
  Office of Science, and their work was performed under the auspices
  of Lawrence Livermore National Laboratory's Contract
  DE-AC52-07NA27344.
\end{acknowledgements}

\bibliographystyle{copernicus}
\bibliography{refs}

% Reviewer comments and responses

\pagebreak

% gmd-2018-52-SC1.txt

\textbf{gmd-2018-52-SC1}

Interactive comment on “Requirements for a global data infrastructure
in support of CMIP6” by Venkatramani Balaji et al.

R. Abernathey
rpa@ldeo.columbia.edu
Received and published: 4 April 2018

Authors: Ryan Abernathey, Naomi Henderson (Lamont Doherty Earth Observatory of
Columbia University), Niall H Robinson, Jacob Tomlinson (Informatics Lab, Met Office,
Exeter), Kevin Paul, Joseph Hamman (National Center for Atmospheric Research), Jiawei Zhuang (School of Engineering and Applied Sciences, Harvard University), Daniel
Rothenberg (ClimaCell, Boston, MA), Matthew Rocklin (Anaconda Inc)...all on behalf
of the Pangeo Project (https://pangeo-data.github.io/)

\begin{enumerate}[label=SC1-\arabic*,leftmargin=*]
\item We commend the WIP for the rigorous and thoughtful assessment of
  the global data infrastructure needed to support CMIP6 and beyond.
  This paper identifies many important challenges related to CMIP data
  replication, provenance, and scientific reproducibility. Absent,
  however, is a discussion of the computational challenges associated
  with the analysis of CMIP datasets and the relationship between data
  archives and computing resources. Our overall recommendation for
  revising the paper is to give more attention to this important
  question. The authors of this comment believe that enabling
  efficient, accessible, scalable computation on CMIP data should
  inform the design of the global infrastructure. Instead of
  encouraging users to download the data to their local systems, we
  should be encouraging users to bring their computing to the data.
  This can be achieved by working more closely with national computing
  centers and by placing CMIP data in cloud storage, where it is
  directly accessible to distributed computing. As recognized in the
  manuscript, many of the most valuable science results from the CMIP
  project come from global comparisons across many models, scenarios,
  and ensemble members. To obtain these results, scientists must run
  analysis on significant fractions of the multi-petabyte CMIP
  archives. As anyone who performs such calculations knows, they
  rarely work on the first try--interactive exploration and
  visualization of the data is a crucial part of the scientific
  process. However, the computing systems deployed for the analysis of
  CMIP data generally fall far short of producing interactive speeds;
  instead researchers wait for weeks to test new ideas (we know this
  from personal experience). Most of these computing systems are what
  the manuscript calls “dark repositories,” mirrors of CMIP data on
  servers and computing clusters owned and managed by individual
  research groups. In addition to disrupting the chain of tracking,
  provenance, and curation (as discussed in the manuscript), dark
  repositories are potentially financially wasteful, since the data is
  transmitted and duplicated over and over just for the purpose of
  exposing it to computation. Scientists must make an up-front
  judgement on which fractions they wish to mirror; they may not even
  use everything they download. In addition, such a priori decisions
  create an insidious pressure to look for "things you expect to see,
  in places you expect to see them." These dark repositories are
  ultimately funded by agencies such as the US National Science
  Foundation and its international counterparts, via equipment
  purchases and technical support staff salaries. No one really knows
  how many dark repositories there are and how much they cost in
  aggregate. Despite the prevalence of dark repositories, users are
  probably frustrated with their performance on terabyte-scale, let
  alone petabyte-scale calculations. A key technical consideration is
  that, on standard servers and workstations, most CMIP-style data
  analysis is heavily I/O bound rather than compute bound i.e. it is
  limited strongly by the rate at which data can be read from storage.
  Fortunately, more scalable ways for climate scientists to interact
  with large datasets are starting to emerge. Intelligent subsetting
  and lazy loading can circumvent the need for bulk downloads.
  Furthermore, when such datasets are placed on distributed storage
  attached directly to distributed computing, the time-to-result for a
  given analysis can be reduced by orders of magnitude, ultimately
  resulting in faster scientific progress. NCAR’s CMIP analysis
  platform is a good example, with CMIP data stored on GLADE (Globally
  Accessible Data Environment), a high performance parallel filesystem
  accessible from the compute nodes of the Cheyenne supercomputer.
  Users with access to this platform are much less likely to want to
  create their own dark repositories, since they enjoy the combination
  of high performance computation and comprehensive data access.
  Although storage on GLADE is expensive compared to a single dark
  repository, it’s probably cheaper than ten dark repositories in
  aggregate.

  While traditional supercomputers can meet some of the data-analysis
  needs of CMIP users, they were not designed for this purpose and are
  probably overkill for it. We believe that an ideal data analytics
  system for these problems has the following properties:

  1. Low administrative hurdles to sign up and log in, even for new,
  junior, or industry users

  2. Easy web access for popular interactive environments like Jupyter
  notebooks

  3. Easy web access on the open internet for automated web services
  and mobile apps


  4. Dynamic and immediate allocation of interactive compute resources
  at modest sizes (hundreds rather than millions of cores) even if
  those sessions may have to grow or shrink during the allocation,
  depending on external use

  5. Cheap costs, sacrificing the high performance network and rich
  CPU/Memory ratio of super-computing centers, and replacing them with
  commodity networking and locally attached storage

  6. Co-location with the relevant datasets

  Data analytics clusters are growing within existing computing
  facilities today that have some (but rarely all) of the properties
  above. Cloud computing, however, is ideally suited to the storage,
  processing, and distribution of extremely large, shared datasets
  today. Both, government-sponsored cloud-style data centers, and the
  commercial cloud (e.g. Amazon Web Services, Google Cloud Platform,
  Microsoft Azure, etc.) merit consideration. Data stored in cloud
  storage is directly accessible from cloud computing instances within
  the same network, providing effectively infinite data bandwidth to
  distributed processing systems. In this paradigm, no data needs to
  be downloaded at all; if the CMIP data were already in cloud
  storage, users would pay only for the compute time they need to do
  their analysis. The cost of hosting 2PB of data on any of the
  commercial cloud providers is roughly \$500K USD per year
  (https://cloud.google.com/products/calculator/id=8ee0d849-a19b-44ab-b5461b0c0dbe775d).
  This is no small sum, but it is likely much less than the collective
  operating budget of the ESGF nodes. The overall financial cost to
  funding agencies might even turn out to be less if individual
  research groups were persuaded to abandon their dark replicas and
  associated local data storage and computation costs in favor of
  cloud computing. Furthermore, commercial cloud providers might also
  provide hosting for free, as they already do for many other
  scientific datasets (e.g. https://aws.amazon.com/public-datasets/,
  https://cloud.google.com/public-datasets/), if they think it will
  bring them customers from academia and industry. Beyond academic
  research, CMIP data hold strong commercial value in sectors such as
  insurance and energy. If CMIP datasets can be liberated from closed
  institutional infrastructure, such consumers can more easily combine
  them with co-located domain specific datasets to gain insights and
  derive economic benefits. NOAA (an agency already contributing model
  development and simulation resources to CMIP) has recently adopted
  such an approach to power their Big Data Project through Cooperative
  Research and Development Agreements and could provide an example for
  future development within the climate science community. The
  scientific payoff from co-locating CMIP data with distributed
  computing resources would be immense, both accelerating
  reproducibility and driving innovation in data analysis
  methodologies--including new machine learning and artificial
  intelligence techniques. But leveraging full advantage of
  distributed systems for analyzing climate data requires more than
  raw hardware; it also requires software which allows climate
  scientists to parallelize their calculations in a simple, efficient
  and transparent way, permitting them to focus on science rather than
  the details of the underlying computing system. A central focus of
  the Pangeo project is to develop these tools on distributed
  computing systems and deploy them on high-impact geoscience
  problems. The building blocks for such software exist, for example,
  in the scientific Python community in the form of packages such as
  xarray (https://xarray.pydata.org), Iris
  (http://scitools.org.uk/iris/), Dask (https://dask.pydata.org), and
  Jupyter (https://jupyter.org/), but require engagement from the
  broader climate science community to reach their full potential for
  our field. We stand ready to work with WCRP and ESGF to help our
  community transition to a cloud-based future.
\end{enumerate}

\begin{answer}
  This is an excellent analysis of the issues involved with the
  transition to the cloud. We have added a brief discussion of the
  \emph{current} status of such efforts, \plref{SC1-1}. We believe
  that while exciting, these efforts are not yet mature enough to
  warrant being mentioned as infrastructural requirements. Indeed, it
  is a shortcoming of our field that ``bringing analysis to the data''
  has been long promised and not yet delivered (witness the history of
  earlier projects such as the G8-funded ExArch project). Many
  of use share Pangeo's vision and look forward to seeing their
  efforts come to fruition.

  We agree also about the point about wasteful duplication of data,
  and the potential for cloud storage and computing to alleviate this
  problem. However, by the ``embracing the dark repository'' approach,
  we hope (as shown in Figure~\ref{fig:dark}) to chart a path
  forward where user communities, at large institutions or consortia
  around particular scientific interests, can build their own shared
  repositories, which can be thought of as private clouds. The most
  wasteful use is for individual scientists in neighboring offices to
  download the same data onto their separate private stores.
\end{answer}

\pagebreak

% gmd-2018-52-RC1.txt

\textbf{gmd-2018-52-RC1}

Interactive comment on “Requirements for a global data infrastructure
in support of CMIP6” by Venkatramani Balaji et al.

Anonymous Referee \#1
Received and published: 23 April 2018

Overview

\begin{enumerate}[label=RC1-Overview-\arabic*,leftmargin=*]
\item This paper reviews the infrastructure requirements needed to
  make CMIP6 successful. There are some attempts at charting a path
  towards the future. Overall, in spite of my numerous specific
  comments below, the paper is well presented with a few notable
  exceptions. My biggest complaint is that after reading the paper, I
  am not sure who the target audience is for this paper. This makes my
  job as a reviewer much harder, since I am guessing at the answer to
  that question. I have assumed that the audience are those who want
  to know something about how the networking/software part of CMIP
  works. This includes some of the modelers and folks in the large
  climate modeling institutions and a subset of the more comp-sci
  oriented users of the CMIP data. If other audiences are in view then
  my review would be very different. This paper is fairly technical.
  My second big picture issue is that references are needed in many,
  many places to either point the reader to supporting documentation
  or to find web sites that explain in more detail what the functions
  are of the various groups/position papers mentioned in the paper.
  Finally, references are also needed to support the statements made
  in the paper. My specific comments below highlights many of the
  missing references.

  \begin{answer}
    We thank the reviewer for a thorough and knowledgeable reading of
    the paper. In the revised text, we have addressed explicitly the
    question of intended audience, \plref{RC1-Overview-1} Re the point
    about references, see below the answer to RC1-15. Several
    additional citations have been added as well.
  \end{answer}
\item Lastly, Section 3.4 needs rewritten. It is very confusing. There
  are lots of recommendations. In places, the language reads like
  these are a requirement. In other places, the prose basically say
  that the recommendations can be ignored. There needs to be some
  priority applied to the discussion. The readers need to know at the
  beginning of the section what is coming -- requirements,
  recommendations, best practices or what. Each item discussed needs
  to be clearly defined in one of the bins -- requirements,
  recommendations, etc. Some parts may be able to be deleted.

  \begin{answer}
    The distinction between findings, requirements, and
    recommendations is made clear now in the introduction,
    \plref{RC1-Overview-2}. The section has been considerably edited
    in response to this comment, and comments below, RC1-20--24.
  \end{answer}
\end{enumerate}

Specific Comments

\begin{enumerate}[label=RC1-\arabic*,leftmargin=*]
\item 1. Page 1, line 11 -- purpose of assigning credit -- This seems
  awkward/backwards to me. The tracking is so that the credit is
  clearly assigned, not the reverse.

  \begin{answer}
    Agreed, awkward wording removed, \plref{RC1-1}
  \end{answer}
\item 2. Page 2, line 6-8 -- A references is needed for this statement.

  \begin{answer}
    Agreed, added, \plref{RC1-2}
  \end{answer}
\item 3. Page 2, line 11 -- capable -- Wrong word. “Available” is a
  better word. There were other climate models available around in the
  world at that time.

  \begin{answer}
    Agreed, \plref{RC1-3}
  \end{answer}
\item 4. Page 2, line 15 -- Add “group” after Manabe.

  \begin{answer}
    Agreed, \plref{RC1-4}
  \end{answer}
\item 5. Page 2, line 16 -- Add “group” after Hansen.

  \begin{answer}
    Agreed, \plref{RC1-5}
  \end{answer}
\item 6. Page 2 Line 17 -- 24 -- The role of AMIP is missing here in the
  formation of CMIP. I agree that the IPCC also played a role, but
  Larry Gates and AMIP was a necessary step to have CMIP formed.

  \begin{answer}
    Agreed, reference added, \plref{RC1-6}
  \end{answer}
\item 7. Page 2, line 23 -- I believe there are now 23 MIPs.

  \begin{answer}
    Agreed. We note 2 new MIPs have been added since the first draft
    of this paper, as well as the canonical citation for CMIP6,
    \cite{ref:eyringetal2016a}, which is used in the text. The new
    wording reflects this evolution, \plref{RC1-7}
  \end{answer}
\item 8. Page 3, line 10-17 -- References for CMIP3 and 5 are missing.

  \begin{answer}
    We believe this is covered by earlier references to
    \cite{ref:eyringetal2016a}.
  \end{answer}
\item 9. Page 5, line 4 -- Reference needed for IPCC.

  \begin{answer}
    Added at first reference to IPCC, \plref{RC1-9}
  \end{answer}
\item 10. Page 7, line 2 -- consumers -- Is “society” a better word
  choice here?

  \begin{answer}
    We believe ``value to society'' of individual datasets is hard to
    assess, but value to actual data users/consumers -- for example by
    citation counts -- is a measurable quantity.
  \end{answer}
  
\item 11. Page 7, line 8 -- Designing -- I think the CMIP Panel
  understands the cost of participating in CMIP since it is mainly
  made up of modelers. It could be argued that some of the new MIP
  chairs in CMIP6 do not understand. Certainly, most users do not
  understand. Reword.

  \begin{answer}
    Reworded, \plref{RC1-11}
  \end{answer}
\item 12. Page 7, line 9 -- Add “data archived in” before CMIP
  experiments.

  \begin{answer}
    Reworded, \plref{RC1-12}
  \end{answer}
\item 13. Page 7, lines 7-10 -- This section is vague. Expand and
  define exactly what is in view here. I assume it includes model
  development, cpu and storage costs, people time and etc. What is in
  view? Exactly what costs are in view?

  \begin{answer}
    Some explanatory text added, \plref{RC1-13}
  \end{answer}
\item 14. Page 7, line 19 -- machine readable experiment design -- This
  needs to be explained here. Page 8, line 14 has a similar problem.
  It needs noted that this is a goal of this effort.

  \begin{answer}
    Some explanatory text added, \plref{RC1-14}
  \end{answer}
\item 15. Page 7, line 29 -- A reference and location is needed for the
  fact sheet.

  \begin{answer}
    In the first draft we used embedded URLs, which were not visible
    by editorial decision on coloured text. All URLs have now been
    made visible as footnotes, including this one, \plref{RC1-15}
  \end{answer}
\item 16. Page 8, line 5 -- Where are these position papers found???
  Are they peer reviewed, citations?

  \begin{answer}
    As noted, these are listed in Appendix~\ref{sec:wip} as noted. The
    citations are there, but as above the embedded URLs were
    invisible. They are now visible in footnotes. The position papers
    are not themselves peer-reviewed, though publicly available for
    comment: this paper is in fact their peer review.
  \end{answer}
\item 17. Page 8, line 13 -- machine readable -- This needs defined.
  Anything stored in a computer is machine readable. . .by definition.
  More is needed.

  \begin{answer}
    It is explained just below, as being encoded in structured text
    documents in XML or JSON format for example.
  \end{answer}
\item 18. Page 10, line 19 -- smaller -- I think “larger” is correct. .
  .nearer to 1. The exponent is larger.

  \begin{answer}
    Explanatory text added clarifying why it is in fact smaller, not
    larger, \plref{RC1-18}
  \end{answer}
\item 19. Page 10, line 24 -- Add “the first part of complexity”
  somewhere near here. The second paragraph starts with the “second
  component of complex” which is confusing given the prose in the
  first paragraph.

  \begin{answer}
    Fixed, \plref{RC1-19}
  \end{answer}
\item 20. Page 11, line 3 -- WIP has recommended -- This seems in
  conflict with line 11 and page 12, line 32. As I note in my general
  comments section, this section is not well written or thought out.
  What message do the authors want to convey to the readers? Rewrite.

  \begin{answer}
    We have considerably rewritten Section~\ref{sec:dvol} for clarity,
    following this reviewer's and others' recommendation.
  \end{answer}
\item 21. Page 11, lines 4-24 -- Regridding -- I understand the Griffies
  papers have a long discussion of the advantages and disadvantages of
  regridding, but a summary of those papers need to be presented here.
  The whole discussion of the disadvantages of regridding is missing
  here.

  \begin{answer}
    Discussion added, \plref{RC3-14}.
  \end{answer}
\item 22. Page 11, lines 4-24 -- Common grid -- So what are the authors
  recommendations for a common grid or regridding? If there are none,
  then delete this discussion to just a summary of the Griffies
  papers.

  \begin{answer}
    The distinction between findings, requirements, and
    recommendations was made explicit in the introduction,
    \plref{RC1-Overview-2}. Furthermore, we have made explicit that
    where there is no consensus, we can but present arguments for and
    against, as this reviewer has requested. We have duly represented
    here the debate around regridding, calendars, and data deflation,
    and noted the lack of consensus. The debate needs to continue, in
    the literature and in other forums of experimental design, until a
    compromise is achieved. We have also described, in
    Section~\ref{sec:pid}, a tracking mechanism to provide data for
    this debate, in terms of what user preferences are with respect to
    these issues.
  \end{answer}
\item 23. Page 11, lines 32-33 -- Again, what is the recommendation? If
  none, what is the justification for keeping the text?

  \begin{answer}
    See above.
  \end{answer}
\item 24. Page 12, lines 4-10 -- What is the recommendation? If any, it
  needs highlighted. Has the WIP surveyed CMIP users in regard to
  these recommendations? I am worried that many users will not be able
  to handle compressed files or shuffled data files.

  \begin{answer}
    See answer above. There has been no explicit survey of users in
    this regard. Shuffling and reinflation are automatic and
    transparent to the user if using netCDF4 libraries.
  \end{answer}
\item 25. Page 12, line 8 -- coupled model -- Define. There are many
  types coupled models in climate. I assume AOGCM and ESMs are in
  view.

  \begin{answer}
    Fixed, \plref{RC1-25}
  \end{answer}
\item 26. Page 12, line 15 -- I do not see what the advantages are of a
  modeling center having this tool. Please explain. The center should
  know its model’s grid and variables to be archived. . ..

  \begin{answer}
    Explanatory text added, \plref{RC1-26} 
  \end{answer}
\item 27. Page 12, line 18 -- Add “compressed” before “data volume”.

  \begin{answer}
    \plref{RC1-27}
  \end{answer}
\item 28. Page 12, line 20 -- Add “current CMIP 3 and 5” before archive
  size.

  \begin{answer}
    \plref{RC1-27}
  \end{answer}
\item 29. Page 12, line 21 -- 25 -- The sentences that start with “The
  more dramatic . . ..” And end with “in years simulated” seems out of
  place and should be moved much earlier.

  \begin{answer}
    Agreed, some lines have been a few paragraphs above,
    \plref{RC1-29a} and \plref{RC1-29b}.
  \end{answer}
\item 30. Page 12, lines 26-27 -- an attempt to impose rational order
  on CMIP5, rather than a qualitative leap” -- What is the unit of
  measure here? Be careful to fully explain this phrase. As is it
  could easily be misused or misunderstood. If CMIP6 is just imposing
  order, why the large expenditure of resources?

  \begin{answer}
    The sentence states that CMIP6's structural innovation
    (DECK+endorsed MIPs) imposes order, not CMIP6 itself. We believe
    this sentence should be allowed to stand.
  \end{answer}
\item 31. Page 12, line 32 -- merely recommendations -- As noted in my
  general comments, this paper needs to be much clearer what is meant
  by “recommendation”.
\item 32. Page 13, fig. 2 caption -- data usage pattern -- It seems to
  show data access, not usage.

  \begin{answer}
    Agreed, caption fixed.
  \end{answer}
\item 33. Page 13, line 4 -- Add “third party” in front of “copies”.
  Also delete rest of sentence after “copies”. It is not clear what is
  meant and seems redundant with first half of sentence.

  \begin{answer}
    We have added ``third party'' as suggested, but believe that the
    reference to the snapshots should be allowed to stand, as they
    were a notable community contribution to CMIP5.
  \end{answer}
\item 34. Page 13, line 16 -- More is needed here. How will a modeling
  center know when somebody is misusing its data? Is their any
  software existing or planned to help a center track its data? If so,
  it needs mentioned here. Furthermore, how can the license change in
  time in this scheme? Many centers make their data public after a
  period of time. It seems that the data files will need to be
  rewritten to change the license agreement. Is this the plan?

  \begin{answer}
    It is not possible to know when someone is using, let alone
    misusing, data, until someone notices and informs the data
    provider. We assume here the reviewer means by ``misuse'', a
    contravention of the license terms. If ``misuse'' is intended to
    mean mis-interpretation, we rely on the journal peer review
    process to prevent that.

    Even if such tracking technologies were available, they would be
    quite intrusive, and quite surely involve privacy violations.
    However, when data is properly cited following the findings
    outlined in Section~\ref{sec:cite}, data providers will be able to
    assess the utility of their data. We believe this will be a
    substantial advance over current practice.

    As regards a center changing the terms of their license after the
    data has been published, that will require the issuance of a fresh
    PID. The terms of use require the user to adhere to the license
    associated with the PID used.
  \end{answer}
\item 35. Page 14, line 1 -- Reference needed (location) of the . .
  ..4.0 International License.

  \begin{answer}
    References visible now, see answer to RC1-15.
  \end{answer}
\item 36. Page 14, line 13 -- Consortium -- Reference, web site?

  \begin{answer}
    References visible now, see answer to RC1-15.
  \end{answer}
\item 37. Page 14, line 28 -- Handle System -- Reference.

  \begin{answer}
    Reference added, \plref{RC1-37}
  \end{answer}
\item 38. Page 15, line 4 -- position paper -- Where is this found?

  \begin{answer}
    References visible now, see answer to RC1-15.
  \end{answer}
\item 39. Page 15, line 11 -- DataCite infrastructure -- Reference and
  location.

  \begin{answer}
    References visible now, see answer to RC1-15.
  \end{answer}
\item 40. Page 15, line 22 -- informally peer reviewed -- This needs
  better defined. Unclear what this is.

  \begin{answer}
    Clarified, \plref{RC1-40}.
  \end{answer}
\item 41. Page 15, line 27 -- collections are static -- How will groups
  correct errors found after the DOI is set? How will corrected data
  be made available? How will users know there are corrections?

  \begin{answer}
    We have clarified the treatment of errors, \plref{RC1-41}. Users
    can discover if the data (PIDs) they are using are superseded
    using the errata service, Section~\ref{sec:errata}.
  \end{answer}
\item 42. Page 16, figure 3 caption -- PID architecture . . . -- PID is
  not found in the figure. How/What things in figure gets a PID? The
  current figure caption should read ``A cartoon of data generation. .
  ..''

  \begin{answer}
    Caption to Figure~\ref{fig:pidarch} updated.
  \end{answer}
\item 43. Page 16, line 5 -- global Handle registry -- Reference, web
  site needed.

  \begin{answer}
    Added above, \plref{RC1-37}
  \end{answer}
\item 44. Page 16, line 9 -- CMIP6 Handle service -- Reference, web site
  location needed

  \begin{answer}
    Added above, \plref{RC1-37}
  \end{answer}
\item 45. Page 16, line 11 -- Add “for all simulation times” after “a
  single experiment”. . . if correct. If not, add details.

  \begin{answer}
    Clarified, \plref{RC1-45}
  \end{answer}
\item 46. Page 16, line 13 -- position paper -- Location?

  \begin{answer}
    References visible now, see answer to RC1-15.
  \end{answer}
\item 47. Page 17, line 1 -- Is there software to generate such a list?
  Seems like in multimodel studies such a list could be very long.
  Will journals publish a long list?

  \begin{answer}
    Indeed, a list of PIDs could be very long. In general, journals
    (including even leading ones such as \emph{Science}) do not count
    supplementary material against page count limits or costs, nor do
    they include them in print versions, so the length should not be
    an issue.

    If the reviewer is asking if the WIP is providing software for
    this purpose, the answer is no. But as the PIDs are in the netCDF
    files, it cannot be seen as difficult for scientists to harvest
    them from the files they use in their research.

    Text unchanged.
  \end{answer}
\item 48. Page 17, line 4 -- RabbitMQ -- Reference needed.

  \begin{answer}
    References visible now, see answer to RC1-15.
  \end{answer}
\item 49. Page 17, line 20 -- CMOR -- Reference and web site needed.

  \begin{answer}
    References visible now, see answer to RC1-15.
  \end{answer}
\item 50. Page 17, line 21 -- PrePARE -- Reference and web site needed.

  \begin{answer}
    References visible now, see answer to RC1-15.
  \end{answer}
\item 51. Page 18, line 4 -- QA nodes -- I assume this is software. As
  written seems like hardware. More is needed.

  \begin{answer}
    It is indeed hardware. Text updated, \plref{RC1-51}.
  \end{answer}
\item 52. Page 19, line 6 -- realms -- Define.

  \begin{answer}
    \plref{RC1-52}
  \end{answer}
\item 53. Page 19, line 7 -- a set of tables -- More is needed or
  delete.

  \begin{answer}
    Added, \plref{RC1-53}
  \end{answer}
\item 54. Page 19, line 13 -- version-controlled code -- Add “software
  that generates versioncontrolled code”. It’s all code. . .

  \begin{answer}
    Clarified, \plref{RC1-54}
  \end{answer}
\item 55. Page 20, line 21 -- embedding -- By whom? Modeler?

  \begin{answer}
    Clarified, \plref{RC1-55}
  \end{answer}
\item 56. Page 20, line 26 -- position paper -- Location?

  \begin{answer}
    References visible now, see answer to RC1-15.
  \end{answer}
\item 57. Page 20, Replication section -- I did not see any way for
  1-off data sets to be issued PIDs. I appreciate that this is hard to
  enforced but the major impact user distribution sites should be
  required to issue PIDs in this framework. Numerically, the impact
  users are the single biggest group using CMIP data. Many of the
  sites serving them, preprocess the model data -- generating new data
  sets, subsets, averages and so forth. These new data sets should not
  have model PIDs, but their own.

  \begin{answer}
    This is an excellent point, and we have added clarifying text,
    \plref{RC1-57}
  \end{answer}
\item 58. Page 21, line 4 -- This statement implies that there are some
  CMIP data sets NOT accessible across ESGF. Is this true? More needed
  here. It is not clear what is meant.

  \begin{answer}
    Clarified, \plref{RC1-58}
  \end{answer}
\item 59. Page 21, line 11 -- ICNWG -- Reference, web site needed.

  \begin{answer}
    References visible now, see answer to RC1-15.
  \end{answer}
\item 60. Page 21, line 13 -- synda -- Reference, web site needed.

  \begin{answer}
    References visible now, see answer to RC1-15.
  \end{answer}
\item 61. Page 22, fig. 7 caption -- CMIP6 replication team -- It says
  CDNOT does this on the previous page. Correct.

  \begin{answer}
    Clarified in the caption to Figure~\ref{fig:replica}, also
    \plref{RC1-61}.
  \end{answer}
\item 62. Page 22, lines 3-6 -- Does this break the data chain (PID and
  etc.)? More needed.

  \begin{answer}
    More explanatory text added, \plref{RC1-62}
  \end{answer}
\item 63. Page 23, Errata section -- Are the replication nodes inside
  or outside of CMIP? This is not clear.

  \begin{answer}
    It is not clear what text the reviewer is referring to, as there
    is no reference to replication nodes in the Errata section.
    Nonetheless, as a general comment, we have attempted to move away
    from the notion of ``inside'' and ``outside'' nodes: for instance,
    \plref{RC1-62}
  \end{answer}
\item 64. Page 24, line 25 -- our data -- Change to “climate” or “CMIP”
  data.

  \begin{answer}
    Corrected, \plref{RC1-64}
  \end{answer}
\end{enumerate}

\pagebreak

% gmd-2018-52-RC2.txt

\textbf{gmd-2018-52-RC2}

Interactive comment on “Requirements for a
global data infrastructure in support of CMIP6” by
Venkatramani Balaji et al.
Anonymous Referee \#2
Received and published: 23 April 2018

General comments

\begin{enumerate}[label=RC2-\arabic*,leftmargin=*]
\item The manuscript provides an overview of WRCP’s Infrastructure
  Panel (WIP) work, discussions and recommendations regarding the
  evolution of CMIP6’ cyberinfrastructure. It discusses some of the
  limitations of the current system, projections for future
  requirements and the rationale for decisions made by the WIP. It
  also describes some of the systems that are being put in place in
  preparation for CMIP6, in particular to better support citations,
  errata and provenance information for datasets and large ensembles,
  as well as managing the increasing volume of information to be
  stored. The paper would benefit from an in-depth editorial review.
  It abuses bullet lists and the level of technical detail varies
  considerably across sections and topics. The result is that although
  interesting and pertinent, the manuscript is at times confusing and
  hard to decipher. I was sometimes left with the impression that the
  paper was composed by copy-pasting sections of various WIP reports.
  The big picture (data-centric system) only really became clear to me
  at the end of a second reading; many of its implications are
  scattered across and not properly merged and highlighted in the
  conclusion. Indeed, the conclusion deserves some love, as at the
  moment it consists in fairly disjointed bullet list items. The
  figures would also benefit from some attention as they apparently
  have been created independently from each other, and their content
  does not always support very well the text around them. Most of my
  suggestions below concern style, as I understand that the manuscript
  has to reflect the WIP’s finding and work, which can’t be modified
  to please reviewers. I think however that the paper should leave
  some room to discuss criticisms made here and elsewhere and possibly
  respond to those. Among these would be the relatively small
  attention given to server-side analytics (raised by another
  referee). I also wonder why the paper does not discuss
  user-feedback? Is this the responsibility of the WIP, ESGF or CDNOT?
  How does the WIP consult users, what do they think of the tools that
  are built and operated for them? The paper makes no mention of
  recommendation concerning the user interface of public facing
  services. Does the priority setting process involves non-IT
  scientific users? Does the WIP include representatives from
  institutions operating dark repositories? Clearly they are prime
  users of CMIP data, yet feel the need to duplicate functionalities,
  and I somehow doubt it is only a matter of bandwidth optimization.
  Other topics not addressed by the paper are software security and
  openaccess, as many of the technical issues that have frustrated
  users and complicated the life of software developers had to do with
  access tokens. I feel the paper would be stronger if it discussed
  the feedback it got from the downstream climate science community
  and used this paper as an opportunity to communicate with it. I
  think there is a need for such a communication exercice after the
  frustrating experience some have had with CMIP5 data access in the
  past.

  \begin{answer}
    The reviewer raises several excellent points, and addressing those
    has considerably clarified the text. To begin with, we have
    (hopefully) addressed some of the stylistic issues, such as the
    ``abuse'' of bullets, point well taken. Second, we hope the text
    makes clear that the WIP has indeed taken the temperature of many
    of the players in this arena (through in-depth consultations, not
    mass email): we note in particular that data \emph{users} alone
    are not the target of this temperature-taking: data providers,
    managers of repositories (official and dark), and users have all
    been taken into account, and indeed are represented on the WIP
    itself. We have restructured the document to provide more context,
    including historical; and considerably rewritten the conclusions
    with more ``love'', we hope.

    Also, as the reviewer notes, the findings here can be challenged
    if they are technically incorrect, but if they reflect the current
    community consensus (or lack thereof), we can but report that in
    this article, which we have done. The distinction between
    findings, requirements, and recommendations is made explicit,
    \plref{RC1-Overview-2} We have strictly followed that nomenclature
    in the subsequent text.
  \end{answer}

  Detailed comments

  Page | Line | Comment
\item 1 7 "data as a commodity in an ecosystem of user" what does this
  mean exactly?

  \begin{answer}
    Clarified, \plref{RC2-2}
  \end{answer}
\item 1 11 dataset-centric: Shouldn’t the objective be for the system
  to be user-centric?

  \begin{answer}
    Of course, the intent is always to be ``user-centric''. In
    practice however, we believe we cannot anticipate all possible
    user needs, as the users of climate data are very diverse, and the
    science continues to evolve. This is why we have tried to
    introduce the notion of a data ecosystem, \plref{RC2-2}.

    The distinction we are trying to make here is that there is no
    giant software infrastructure that is itself a single point of
    failure. Once users have datasets in their hands, or even their
    PIDs, they can continue to perform data transactions peer to peer
    even if, for instance, some key ESGF nodes go down.

    This point is repeatedly brought up throughout the text, and here
    in the abstract in the phrase ``less prone to systemic failure''.
    No changes in response to this reviewer comment.
  \end{answer}
\item 2 9 prescient: maybe a bit strong

  \begin{answer}
    \plref{RC2-4}
  \end{answer}
\item 2 15 3 -> three. As a general rule, spell numbers < 10

  \begin{answer}
    \plref{RC2-5}
  \end{answer}
\item 2 18 5 -> five

  \begin{answer}
    \plref{RC2-6}, \plref{RC2-6b}
  \end{answer}
\item 2 18 "formalized" used in last sentence and sentence is unclear.
  Mix of historical and current (DECK) denominations is confusing.

  \begin{answer}
    reworded, reference added, \plref{RC2-7}
  \end{answer}
\item 3 6 in in Figure 1

  \begin{answer}
    \plref{RC2-8}
  \end{answer}
\item 3 6 (some of) remove parentheses

  \begin{answer}
    \plref{RC2-12}
  \end{answer}
\item 3 8 Is the ESGF a "component". It looks to me as a loosely
  structured organization, with a "soft leadership", which indeed
  poses a number of challenges in terms of planning and delivery of
  operational software. This is possibly out of scope for this paper,
  but consider adding a paragraph somewhere in the paper about how
  ESGF organizes to implement WIP recommendations and some of the
  challenges it faces.

  \begin{answer}
    Replaced ``component'' with ``artifact'', \plref{RC2-10}. We agree
    that the ESGF response to WIP requirements is out of scope for
    this paper, and a separate paper on the ESGF itself is warranted,
    once the software stack is finalized, and the system operational.
  \end{answer}
\item 3 12 upon , a proposal

  \begin{answer}
    \plref{RC2-11}
  \end{answer}
\item 4 Figure 1: There is a site that looks to be in James Bay. Also
  is it really necessary to include personal contact email? This is
  something that can get outdated very fast.

  \begin{answer}
    A more appropriate figure has been substituted, see
    Figure~\ref{fig:esgf} and \plref{RC2-12}. Here the nodes are
    mapped to their geographic locations rather than relative to
    national boundaries.
  \end{answer}
\item 5 6 It’s not clear to what "which are summarized here" make
  reference to, "fundamental changes" or the "evolving scientific and
  operational requirements"?

  \begin{answer}
    The clunky phrasing has been removed in the rewrite of
    Section~\ref{sec:infra-principles}.
  \end{answer}
\item 5 7 The presentation is a bit awkward here, with a numbered list
  nesting a bullet list. I feel that this could all be written in text
  form. Also, the text suggests that the following items are
  "changes", but some of the opening statements are not.

  \begin{answer}
    We thank the reviewer for this useful guidance, and the entire
    Section~\ref{sec:infra-principles} has been rewritten as
    suggested, without bullets. Also, re ``changes'', \plref{RC2-14}
  \end{answer}
\item 6 9 review sentence syntax, second clause seems incomplete.
  Again, the bullet format feels innapropriate for dense and elaborate
  content.

  \begin{answer}
    Entire section rewritten as noted above.
  \end{answer}
\item 6 21 The first bullet is the context, and the second the
  requirement. Please maintain some uniformity in the organization of
  ideas.

  \begin{answer}
    Bullets removed, see \plref{RC2-16}
  \end{answer}
\item 7 11 Idem

  \begin{answer}
    Bullets removed, see \plref{RC2-17}
  \end{answer}
\item 8 15 The data request concept is not properly introduced. Please
  clarify what it is and what purpose it is intended to serve before
  providing implementation details.

  \begin{answer}
    Context provided, \plref{RC2-18}
  \end{answer}
\item 8 16 I feel that the level of details given on Data Requests far
  exceeds that of other sections. Who are the intended users? Data
  managers or analysts? Is the level of detail really relevant to this
  paper? Frankly, I read it a couple of times and I still don’t
  understand the role it plays.

  \begin{answer}
    We have rewritten Section~\ref{sec:data-request} at an appropriate
    level of detail, and hope, with the added context, its key role is
    now readily understood.
  \end{answer}
\item 11 3 If I understand correctly, the single most important factor
  in the growth of data volume between CMIP3 and 5 is the number of
  variables that are archived. Yet, this issue does not appear to be
  formally addressed by the WIP as a volume problem further down in
  the text. At the moment, my understanding is that data is saved
  using the 1-file-per-variable approach. With hundreds of variables
  to probably co-vary in time and space, I’m guessing there might be
  compression benefits in storing multiple variables in the same file.

  \begin{answer}
    The data volume discussion has been rewritten, see
    Section~\ref{sec:dvol}. As regards the second point, it is no
    doubt true that many of the variables exhibit considerable
    covariance, and are not statistically independent. But this
    remains still a matter for analysis and discovery. The current
    1-variable-file remains a useful unit of analysis, a compromise
    for most users between too large files and too many files. Future
    infrastructure may indeed move in other directions based on the
    outcomes of CMIP6, and indeed POSIX ``files'' may themselves
    become obsolete, under certain technological evolutionary
    pathways currently at the cutting edge. We have added some
    discussion of these issues in the Conclusion.
  \end{answer}
\item 11 4 The use of a numbered list here makes little sense.

  \begin{answer}
    List removed, \plref{RC2-21}
  \end{answer}
\item 11 4 Please start the paragraph with the recommendation itself.
  Same suggestion applies to second recommendation.

  \begin{answer}
    Section~\ref{sec:dvol} has been considerably rewritten, and the
    recommendations restated at the end of the section.
  \end{answer}
\item 11 13 Is the reference to the name of the actual python file
  really necessary? I suggest putting links to tools and software in
  appendix B.

  \begin{answer}
    We have removed the excess detail in the rewrite of
    Section~\ref{sec:dvol}, \plref{RC2-23}.
  \end{answer}
\item 12 20 CMIP archive size. Are you referring to CMIP5? Please
  clarify.

  \begin{answer}
    \plref{RC1-28}
  \end{answer}
\item 12 21 Sentence is confusing : "same causes, but with a much
  larger change"

  \begin{answer}
    Reworded, \plref{RC1-29b}
  \end{answer}
\item 13 Fig 2. Why "!" after local cache ?

  \begin{answer}
    Gone, see caption to Figure~\ref{fig:dark}.
  \end{answer}
\item 13 14 Is that really ``embracing'' the dark repository model? I
  believe embracing that model would entail something a lot more
  ambitious such as a P2P network between official and dark repos that
  lets ESGF leverage dark repo to replicate and disseminate data. This
  is discussed later with synda (as far as I understand), but would
  deserve discussion here.

  \begin{answer}
    Clarifying text, and connection to replication discussion added,
    \plref{RC2-27}. Replication is peer-to-peer in this system but
    based on units of atomic datasets, not packets, as in say
    BitTorrent. This does not preclude future development of P2P
    replication at the packet level.
  \end{answer}
\item 13 15 Review syntax.

  \begin{answer}
    As this is a quote from another document, it doesn't seem
    appropriate to change the syntax. The meaning seems fairly clear
    to us. The editors should let us know if they disagree.
  \end{answer}
\item 13 18 I don’t understand what this sentence means and how it
  relates to the preceding text.

  \begin{answer}
    Some of this text is indeed out of place, rewritten.
    \plref{RC2-29}. Some of the text is displaced to a discussion of
    the role of cloud analysis platforms in reducing data movement,
    \plref{SC1-1}.
  \end{answer}
\item 13 20 Idem.

  \begin{answer}
    With the changes, we believe there is now continuity in the
    licensing discussion, \plref{RC2-30}.
  \end{answer}
\item 13 26 Please define "handles". Figure 4 Who issues the PID? The
  data producer? This is only discussed later on page 18. I think it
  should be explained earlier.

  \begin{answer}
    Reference for Handles added, \plref{RC1-37}. PID issuance
    introduced here, \plref{RC2-31}.
  \end{answer}
\item 20 17 Close parenthesis

  \begin{answer}
    \plref{RC2-32}
  \end{answer}
\item 21 5 Item 4 in section 2 only discusses model evaluation, not
  general data analysis.

  \begin{answer}
    This section has been generalized as suggested, \plref{SC1-1}.
  \end{answer}
\item Figure 7 It’s not clear what this figure adds
  to the explanation.

  \begin{answer}
    While we encourage ad-hoc replication, we wish to also underline
    the concerted effort to make sure high-value data is not
    repeatedly moved across geographic domains. This figure
    illustrates the efforts to coordinate replica nodes with
    sufficient storage, as well as the involvement of the network
    provisioners (ICNWG). We believe the figure should stay.
  \end{answer}
\item 24 24 Bullet list with no proper introduction. Please write a
  proper conclusion.

  \begin{answer}
    Section~\ref{sec:summary} has been rewritten following reviewer's
    suggestion.
  \end{answer}
\item 25 8 Is that really the message you want to end with? I suggest
  ending with an invitation to the climate science commnuity to
  provide feedback and suggestions, and generally get involved in the
  WIP’s activities.

  \begin{answer}
    See answer above.
  \end{answer}
\end{enumerate}

\pagebreak

% gmd-2018-52-RC3.txt

\textbf{gmd-2018-52-RC3}

Interactive comment on “Requirements for a
global data infrastructure in support of CMIP6” by
Venkatramani Balaji et al.
Anonymous Referee \#3
Received and published: 8 May 2018

The paper describes the challenges for the global data infrastructure
needed to support the ongoing efforts of the climate modelling
community that are organised in the CMIP enterprise. The material
presented is of great importance and should be published. However, its
presentation does not meet the requirements of a journal article and
requires major revision. There are two major issues that need to be
addressed as the paper gets rewritten:

\begin{enumerate}[label=RC3-\arabic*,leftmargin=*]
\item 1. The paper is clearly the result of a lot of work within the
  WGCM Infrastructure Panel. Acknowledging this is important, but the
  paper completely goes overboard and as a result reads like a report
  to some steering committee, rather than a journal article. I counted
  no less than 46 (and I am sure I missed some) occurrences of
  statements like “The WIP recommends. . .,” “The WIP did. . .” or
  “Based on what the WIP thinks . . .”. This is simply not the style
  of a paper. I recommend removing all these references and telling us
  what the authors of this paper think. I realize they are the WIP,
  but the reader does not need to be told this every other paragraph.
  I suggest putting a clear statement that the suggestions of this
  paper are the result of deliberations by what is likely a temporary
  body in the long run, the WIP, and then present what are hopefully
  not temporary conclusions for the infrastructure needs. I also
  suggest avoiding repeated statements that more detail can be found
  in WIP reports. This can be said once and the reports listed in the
  Appendix, as is the case.

  \begin{answer}
    We thank the reviewer for their thorough and candid review. This
    stylistic recommendation has been followed throughout in the
    revision of the text (too many instances to call out here...) and
    we believe has greatly enhanced its readability.
  \end{answer}
\item 2. Perhaps the more important question I struggled with is who the
  intended audience for this paper is, which will define its purpose
  and then structure. If it is scientists and users, the paper needs
  to significantly cut down on jargon (see minor comments below). If
  it is infrastructure communities outside climate, then this should
  be written as an example for what challenges the climate community
  is facing and what it is doing about it, so that others can learn
  from it. Or is it the modelling centres to instruct them on new
  procedures and tools? In that case, a paper is unlikely needed as
  they can be sent an email with the detailed position papers! At the
  moment, neither community will benefit from this paper as it isn’t
  clear what it is trying to achieve. I realize that this is harder to
  solve than issue 1, but it is important to know this before
  rewriting the paper begins. Once it is clear, the goal should be
  stated in the introduction.

  \begin{answer}
    This is a good point raised by RC1 as well, and we have provided
    some context, \plref{RC1-Overview-1}. The text and conclusions
    have been modified as well to make clear the audience and intent.
    A new section on Historical Context has been added,
    Section~\ref{sec:history}.
  \end{answer}

  More minor but often typical issues in chronological ordere
\item Page 2 Line 17 -- The statement that by the FAR of the IPCC
  modelling inter comparisons were formalised is untrue. The first
  formal model inter comparison was AMIP and was reported in 1992.
  (Gates, W. L., 1992: AMIP: The Atmospheric Model Intercomparison
  Project. Bull. Amer. Meteor. Soc, 1962--1970,
  doi:10.1175/1520-047773.12.1962.) CMIP started only after that.
  Please correct this

  \begin{answer}
    Agreed, reference added, \plref{RC1-6}.
  \end{answer}
\item Page 2, Line 18 -- Please cite the appropriate paper when
  referring to the DECK

  \begin{answer}
    Citation added, \plref{RC2-7}
  \end{answer}
\item Page 4, Figure 1 -- Most dots on this figure are in the city the
  node is located. The one in Australia is in the middle of the
  desert. Canberra is not. Please correct.

  \begin{answer}
    See reply to RC2-12.
  \end{answer}
\item Page 8, Line 17 -- What is “The data request”. This needs an
  introductory sentence as only people in the know will know. Same
  line: What is the “DREQ” tool. This is an example for the frequent
  jargon with no explanation. Please be more careful as not every
  reader will already know these acronyms.

  \begin{answer}
    Good point, and we have added context, \plref{RC2-18}. Some
    unneeded jargon removed.
  \end{answer}
\item Page 8, Line 28 -- The sentence about the database allowing MIPs
  to do things is another example for jargon. It means nothing to
  someone who doesn’t already know all this. Please explain it better.
  For instance, highlight that different MIPs will request different
  variables, but some will be common. You can’t assume the reader to
  be a CMIP expert and if you do, why write this paper?

  \begin{answer}
    Section~\ref{sec:data-request} has now been rewritten at what we
    hope is an appropriate level of detail and context.
  \end{answer}
\item Page 9, lines 1-3 -- This list is very confusing and requires
  more context.

  \begin{answer}
    Addressed in the revised Section~\ref{sec:data-request}, bulleted
    lists removed.
  \end{answer}
\item Page 9, lines 16-20: A single paragraph does not deserve it’s own
  subsection. Please correct. The last sentence is another example for
  a sentence from a report that makes little to no sense to an
  independent reader. Please remove those as you rewrite the paper.

  \begin{answer}
    While this section is indeed quite short, the input4MIPs and
    obs4MIPs efforts, and their coherence with the overall data
    design, is an important element and we believe this point will be
    lost if it is buried in a paragraph somewhere. This point is
    highlighted and the language on versioning clarified,
    \plref{RC3-9}
  \end{answer}
\item Page 10, Line 6 -- The statement on increasing data volumes
  overstates the case if its is not put into context. The Large Hadron
  Collider produces vastly larger data volumes than any set of climate
  models ever will! It is important to clarify that the challenge is
  that the data is both produced and used in a distributed network. If
  one place with all the resources needed ran all the CMIP runs from
  all the models, archiving them would be simple! Distributing them
  might still be challenign though!

  \begin{answer}
    Context added, \plref{RC3-10}
  \end{answer}
\item Page 10, lines 28-29 -- What do you mean with “appear to have
  grown”. Has it or not?

  \begin{answer}
    \plref{RC3-11}
  \end{answer}
\item Page 11, line 11 -- What is the “CMIP6 Output grid guidance
  document”? If you use it, you need to provide a reference/link to
  it.

  \begin{answer}
    References visible now, see answer to RC1-15.
  \end{answer}
\item Page 11, line 16-18 -- To an outsider to the climate community
  this appears insane! There is only one real calendar and it has to
  do with the Earth going around the sun in a certain unit of time. It
  would be worth commenting on the future of this, as it implies a
  “laziness” in the climate community to do something simple (I
  understand it is not that simple).

  \begin{answer}
    Good point :-). An explanation of the calendar issue for
    ``outsiders'' has been added, \plref{RC3-13}.
  \end{answer}
\item Page 11, line 23-24 -- Again, to an outsider this sounds strange.
  How can infrastructure that does relatively straightforward analysis
  be overburdened? Isn’t that because the way this is funded is
  inadequate. If you agree, isn’t important to point this out in this
  paper about the future?

  \begin{answer}
    Regridding of data is burdensome for many reasons: we have more
    explicitly pointed out this out now earlier in this section,
    \plref{RC3-14}. We have added a discussion in the Conclusion
    section about the funding constraints.
  \end{answer}
\item Page 11, line 25 -- By now I had now idea that there were two
  issues. Please remove this first and second bit. The first issue was
  several already!

  \begin{answer}
    The numbering has been removed, \plref{RC2-21}.
  \end{answer}
\item Page 12, line 5 -- If the results are public, please cite where
  and how the reader can access them.

  \begin{answer}
    References visible now, see answer to RC1-15.
  \end{answer}
\item Page 12, line 14 -- Where is the WIP website? Please add a link.

  \begin{answer}
    References visible now, see answer to RC1-15.
  \end{answer}
\item Page 12, line 29-30 -- Jargon. We don’t know what a Tier1 node
  is, let alone that it has a manager. Explain or remove!

  \begin{answer}
    \plref{RC3-18}
  \end{answer}
\item Page 14, line 2 -- PCMDI website -- please cite properly by adding
  the link.

  \begin{answer}
    References visible now, see answer to RC1-15.
  \end{answer}
\item Page 14, line 20 -- O(10ˆ6), 10ˆ6 what? Add units.

  \begin{answer}
    It is a dataset count as stated, no units.
  \end{answer}
\item Page 20, line 26 -- A good example of overdoing the WIP(ping).
  The reader does not care where the replication strategy is covered.
  They want to know what there authors of this paper have to say about
  replication.

  \begin{answer}
    \plref{RC3-21}. In general there is much less WIPping in the new
    draft, see answer to RC3-1.
  \end{answer}
\item Page 21, line 21 -- Jargon. What is the CDNOT group? Please
  explain.

  \begin{answer}
    The CDNOT was introduced above, \plref{RC3-22}.
  \end{answer}
\item Page 23, line 21 -- There seems to be only one subsection, so why
  have it? Please remove the heading.

  \begin{answer}
    The Errata section is related to versioning but an important
    independent piece, deserving of its own section, in our
    estimation.
  \end{answer}
\item Section 8 -- I was disappointed by this section, as I was
  expecting a summary of the main challenges and recommendations for
  solutions. I feel some of the challenges need to be spelled out
  here. For instance, funding for the activities described here is
  pretty ad-hoc. This is disturbing given the attention the world pays
  to the data sets in question. Is there something here to discuss?
  Can the world continue to scramble its way through this? Thoughts?
  Other big issues: Doing more and more in CMIP (more models, more
  experiments, more users) cannot be sustained unless investment into
  the enterprise gets better coordinated -- any role for international
  organisations to help with this? Many of the issues are discussed
  are the result of accepting the status quo in distributed climate
  modelling. Should we? Are there more sensible alternatives?

  \begin{answer}
    Section~\ref{sec:summary} has been considerably rewritten. In it
    we have mentioned that prior panels including at the US National
    Academies level, have indeed made this case, but so far to no
    avail. We have explained how the new dataset-centric design is
    indeed intended to reduce systemic risk due to infrastructure
    failure, and allows for a scalable system that is sized at a level
    appropriate to available resources.
  \end{answer}
\item Appendix A -- Might be nice to add links to each of these
  reports (or the main page where they can all be found).

  \begin{answer}
    References visible now, see answer to RC1-15.
  \end{answer}
\end{enumerate}
\end{document}
